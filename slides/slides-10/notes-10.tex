\documentclass{article}

\usepackage[utf8]{inputenc}
\usepackage{amsthm}
\newtheorem{definition}{Definition}
\newtheorem{lemma}{Lemma}
\newtheorem{proposition}{Proposition}
\newtheorem{theorem}{Theorem}
\newtheorem{corollary}{Corollary}

% \beamertemplatenavigationsymbolsempty
% \setbeamertemplate{footline}{
% 	\raisebox{0.5em}{\makebox[\paperwidth-0.5em][r]{
% 		\color{gray}\scriptsize\insertframenumber}}}

\usepackage{mathtools}
\usepackage{tikz}
\usetikzlibrary{cd, shapes, tikzmark}
\def\pathToRoot{../../}\usepackage{nag}
\makeatletter
\@ifclassloaded{beamer}{}
{\usepackage[small,compact]{titlesec}}
\makeatother
\usepackage[utf8]{inputenc}
\usepackage[T1]{fontenc}
\usepackage{lmodern}
\usepackage{color}
\usepackage{parskip}
\usepackage{needspace}
\usepackage{microtype}
\usepackage{mathtools}
\usepackage{xifthen}
\usepackage{xpatch}
\usepackage{enumitem}
\usepackage{mdwlist}
\usepackage{bussproofs}
\EnableBpAbbreviations
\usepackage{tabu}
\usepackage{amssymb}
\usepackage{amsmath}
\usepackage{amsthm}
%grober hack, der den groben hack von parskip bei den amsthm sachen korrigiert
\begingroup
    \makeatletter
       \@for\theoremstyle:=definition,remark,plain\do{%
            \expandafter\g@addto@macro\csname th@\theoremstyle\endcsname{%
                        \addtolength\thm@preskip\parskip
             }%
        }
\endgroup
\usepackage[UKenglish]{babel}
\usepackage{xparse}
\usepackage{adjustbox}
\usepackage{geometry}
\usepackage{booktabs}
\usepackage{multicol}
\usepackage{soul}
\usepackage{calc}
\usepackage{textcase}
\usepackage{stmaryrd}
\usepackage{marvosym}
\usepackage{wasysym}
\usepackage{pifont}
\newcommand{\cmark}{\ding{51}}
\newcommand{\xmark}{\ding{55}}
\usepackage{tikz}
\usetikzlibrary{trees, backgrounds, shapes, chains, decorations.text, decorations.pathreplacing, circuits.logic.IEC, patterns, matrix}
\usepackage{tikz-qtree}
\usepackage{tikzsymbols}
\usepackage{fancyvrb}
\usepackage{fancyhdr}
\usepackage{verbatim}
\usepackage[framemethod=tikz]{mdframed}
\usepackage{lastpage}
\usepackage{pgfpages}
\usepackage{csquotes}
\usepackage{longtable}
\usepackage{ragged2e}
%\usepackage{stackengine}
\usepackage{censor}
\usepackage{expl3}
\usepackage{multirow}
\usepackage{hyperref}
\usepackage{environ}


% Package for Cateogry diagrams:

\usepackage{tikz-cd}



\ifcsdef{labelenumi}{
\renewcommand{\labelenumi}{(\alph{enumi})}
\renewcommand{\labelenumii}{(\roman{enumii})}
}{}

\input{\pathToRoot headers/definitions}



\tikzset{
    normal/.style={draw, semithick},
    n/.style={style=normal, circle, inner sep=1mm, minimum size=8mm},
    l/.style={style=normal, rounded corners=1mm, inner sep=1mm, minimum size=6mm},
    e/.style={style=normal, shorten >=1mm, shorten <=1mm, ->, >=stealth},
    syntax/.style={style=normal, ellipse, minimum height=6mm, minimum width=8mm}, % nodes in syntax trees
    inner/.style={style=normal, minimum size=4mm}, % inner leaves or root in normal trees
    leaf/.style={style=normal, circle, minimum size=4mm}, % leaves in normal trees
    te/.style={style=normal}, % edges in a tree
    be/.style={style=e, dashed} % binding edge
}

\newcommand{\syntaxtree}[1]{ % DEPRECATED - use tikzsyntaxtree
    \begin{tikzpicture}[baseline=(current bounding box.north)]
        \tikzset{grow=down}
        \tikzset{every node/.style={syntax}}
        \tikzset{edge from parent/.style=
            {te,
                edge from parent path={(\tikzparentnode) -- (\tikzchildnode)}}}
        \Tree #1
    \end{tikzpicture}
}

\newenvironment{tikzsyntaxtree}[1][]{
    \begin{tikzpicture}[baseline=(current bounding box.north), #1]
    \tikzset{grow=down}
    \tikzset{every tree node/.style={syntax}}
    \tikzset{edge from parent/.style={te, edge from parent path={(\tikzparentnode) -- (\tikzchildnode)}}}
}{
    \end{tikzpicture}
}


\newcommand{\DisplayScaledProof}{\maxsizebox{\linewidth}{!}{\DisplayProof}}
\newcommand{\DisplayTopProof}{\adjustbox{valign=t}{\DisplayProof}}
\newcommand{\DisplayScaledTopProof}{\adjustbox{valign=t}{\maxsizebox{\linewidth}{!}{\DisplayProof}}}


\newcolumntype{P}[1]{>{\RaggedRight\hspace{0pt}}p{#1}}

\newenvironment{prooftable}
{
    \begin{longtable}{>{\footnotesize}p{0.33\textwidth}>{\footnotesize}p{0.33\textwidth}|>{\footnotesize}P{0.15\textwidth}}
    \normalsize Textbeweis & \normalsize Erklärungen & \normalsize Schlussregel\\\hline
    \endhead
}
{
    \end{longtable}
}


\theoremstyle{definition}
\newtheorem*{definition*}{Definition} % Definition ohne Nummer
\newtheorem*{inferenceRule*}{Schlussregel}


\title{Presheaves, Representables and\\the Yoneda Lemma}
\author{Dominik Wagner}
%\date{June 7, 2017}

\begin{document}

\section{Introduction}
In this chapter we prove the following three propositions, which we mentioned earlier but failed to prove directly:
\begin{itemize}
\item If $\cat A$ is cartesian closed and has coproducts, then for all $A,B,C\in\cat A$
  \begin{align*}
    (A\times B)+(A\times C)\iso A\times (B+C);
  \end{align*}
\item Left adjoints are unique up to isomorphisms;
\item For any small category $\cat A$, the category of presheaves $[\cat A^\op,\Set]$ is cartesian closed.
\end{itemize}
Along the way, we prove a very important technical lemma called ``Yoneda lemma''. One of its consequences is that two obects are isomorphic if and only if they ``look the same from all other objects''.

In order to get a first understanding what this is supposed to mean we consider the example of a partially order set $(P,\leq)$. Let $p,p'\in P$ be arbitrary. Then we clearly have 
\begin{align*}
  (p\leq p')\iff (\forall q\in P\ldotp q\leq p\rightarrow q\leq p')
\end{align*}
or in other terms
\begin{align*}
  (p\leq p')\iff \downarrow(p)=\{q\in P\mid q\leq p\}\subseteq \{q\in P\mid q\leq p'\}=\downarrow(p'). *
\end{align*}

How can we lift this to arbitrary categories? We can do so by noting that the two sets correspond to the maps into $p$ and $q$ respectively. However, since maps are not unique in general categories we have to do a little bit more work. Recall the following special functor introduced in \ref[?]: 

\begin{definition}[Contravariant Representable Functor]
  Let $\cat A$ be a locally small category and $A\in \cat A$. The \emph{contravariant representable functor} of $A$
  \begin{align*}
    H_A=\cat A(-,A)\from \cat A^\op\to \Set
  \end{align*}
  is defined by
  \begin{itemize}
  \item $H_A(B)=\cat A(B,A)$ for objects $B\in\cat A$ and
  \item for maps $B'\xrightarrow{f} B$:
    \begin{align*}
      H_A(f)=-\of f\from \cat A(B,A)&\to \cat A(B',A)\\
      g &\mapsto g\of f\\\\
      \xymatrix{B' \ar[r]^{f} & B \ar[r]^{g} & A }
    \end{align*}
  \end{itemize}
\end{definition}

Clearly, for each $p$ in a preorder regarded as a category $\downarrow(p)$ is essentially the same as $H_p$. Furthermore, inclusion are essentially natural transformations. Then, * corresponds to the fact that the following functor is full and faithful:

\begin{definition}[Yoneda Embedding]
  Let $\cat A$ be locally small. The \emph{Yoneda embedding} of $\cat A$ is the functor
  \begin{align*}
    H_-\from \cat A\to[\cat A^\op,\Set]
  \end{align*}
  defined by
  \begin{itemize}
  \item $H_-(A)=H_A$ for objects $A\in\cat A$ and
  \item for maps $A\xrightarrow{f}A'$, $H_-(f)$ is the natural transformation defined for $B\in \cat A$ by
    \begin{align*}
      H_A(B)=\cat A(B,A)&\to H_{A'}(B)=\cat A(B,A')\\
      g&\mapsto  f\of g **\\
    \end{align*}
  \end{itemize}
  \vspace{-0.8cm}
  \includegraphics{nattrans.pdf}
\end{definition}
Basically, for preorders ** constructs a proof from $p\leq p'$ that for all $q\leq p$ also $q\leq p'$ holds.

In what follows we prove that this is indeed an embedding, i.e. full and faithful.

\section{Yoneda Lemma}
To do so we state the following technical lemma due to Yoneda:
\begin{lemma}[Yoneda]
  Let $\cat A$ be locally small. Then
  \begin{align*}
    [\cat A^\op,\Set](H_A,F)\iso F(A)
  \end{align*}
  naturally in $A\in\cat A$ and $F\in [\cat A^\op,\Set]$. That is for each $B\xrightarrow{f}A$ in $\cat A$ and for each natural transformation $\theta \from F\to F'$ the following commute
  \begin{align*}
    \xymatrix{
    [\cat A^\op,\Set](H_A,F) \ar[r]^{-\of H_f} & [\cat A^\op,\Set](H_B,F) \\
    F(A) \ar[r]^{F(f)} \ar[u]^{\iso}& F(B) \ar[u]_{\iso}
                         } 
  \end{align*}
  \begin{align*}
    \xymatrix{
    [\cat A^\op,\Set](H_A,F) \ar[r]^{\theta\of -} & [\cat A^\op,\Set](H_A,F') \\
    F(A) \ar[r]^{\theta_A} \ar[u]^{\iso} & F'(A) \ar[u]^{\iso}
                             } 
  \end{align*}
\end{lemma}
It is going to help us understand what the arrows $F_A\to F_{A'}$ are (in the presheaf category). More generally, it states that an element of $F(A)$ ``is the same'' as a natural transformation $H_A\to F$.
\begin{proof}
  In order to show the isomorphism in $\Set$, which is supposed to be natural in both $A$ and $F$, we can proceed by:
  \begin{enumerate}
  \item Define a function $(\widehat{-})\from[\cat A^\op,\Set](H_A,F)\to F(A)$.
  \item Define a function $(\widetilde{-})\from F(A)\to[\cat A^\op,\Set](H_A,F)$.
  \item Show that $(\widehat{\widetilde{-}})=1_{F(A)}$.
  \item Show that $(\widetilde{\widehat{-}})=1_{[\cat A^\op,\Set](H_A,F)}$.
  \item Show that $(\widehat{-})$ is natural in $A$.
  \item Show that $(\widehat{-})$ is natural in $F$.
  \end{enumerate}
  It turns out that the definitions of $(\widehat{-})$ and $(\widetilde{-})$ are straight-forward as there is only one canonical choice that can possibly be made:
  \begin{enumerate}
  \item Let $\alpha\from H_A\to F$ be arbitrary. In order to define $(\widehat{-})\from[\cat A^\op,\Set](H_A,F)\to F(A)$, we have to come up with an element $\widehat\alpha\in F(A)$. However, $\alpha$ gives us a function $\alpha_A\from H_A(A)\to F(A)$ in $\Set$. Hence, we only have to come up with an element of $H_A(A)=\cat A(A,A)$ to obtain an element of $F(A)$ and we can simply choose $1_A$. Thus, we define $\widehat\alpha=\alpha_A(1_A)$.
  \item Let $a\in F(A)$. This time, we have to define a natural transformation $\widetilde a\from H_A\to F$, i.e. for each $B\in\cat A$ we have to define a function $\widetilde a_B\from H_A(B)\to F(B)$ in $\Set$ and show that the naturality condition is satisfied.

    In order to define $\tilde a_B$, let $f\in H_A(B)=\cat A(B,A)$ be arbitrary. We have to construct an element of $F(B)$. However, we have a functor $F\from\cat A^\op\to\Set$ and $F(f)\from F(A)\to F(B)$. This is perfect since we already have $a\in F(A)$. Thus, $F(f)(a)\in F(A)$ and we can define 
    \begin{align*}
      \widetilde a_B(f)=F(f)(a).
    \end{align*}

    We still have to check naturality, i.e. that for any $B'\xrightarrow{g}B$ the following commutes:
    \begin{align*}
    \xymatrix{
    F_A(B)=\cat A(B,A) \ar[r]^{H_A(g)=-\of g} \ar[d]^{\widetilde a_B} & F_A(B')=\cat A(B',A) \ar[d]_{\widetilde a_B'} \\
    F(B) \ar[r]^{F(g)} & F(B').
                         } 
  \end{align*}
  Since this square is in $\Set$, it suffices to check $(\widetilde a_B'\of H_A(g))(f)=(F(g)\of\widetilde a_B)(f)$ for every $f\in F_A(B)=\cat A(B,A)$. Indeed, this holds:
  \begin{align*}
    (\widetilde a_B'\of H_A(g))(f)&=(F(f\of g))(a)\\
    &=(F(g)\of F(f))(a)&F \text{ contravariant functor}\\
    &=F(g)(F(f)(a))\\
    &=(F(g)\of\widetilde a_B)(f).
  \end{align*}
\item Let $a\in F(A)$. Clearly, we get $\widehat{\widetilde a}=a$:
  \begin{align*}
    \widehat{\widetilde a}=\widetilde a_A(1_A)=F(1_A)(a)=1_{F(A)}(a)=a.
  \end{align*}
\item[(d)-(f)] The remaining proof obligations are quite straight-forward to show and are thus left as an exercise.
  \end{enumerate}
\end{proof}

Now, it is easy to show that $H_-$ is full and faithful by instantiating $F$ with a covariant representable functor.
\begin{theorem}
  The Yoneda embedding $H_-\from \cat A\to[\cat A^\op,\Set]$ is full and faithful.
\end{theorem}
\begin{proof}
  We have 
  \begin{align*}
    \widehat H_f=(H_f)_A(1_A)=f\of 1_A=f.
  \end{align*}
  Hence,
  \begin{align*}
    \cat A(A,A')&\to[\cat A^\op,\Set](H_A,H_A')\\
    f&\mapsto H_f
  \end{align*}
  and
  \begin{align*}
    (\widetilde{-})\from F(A)\to[\cat A^\op,\Set](H_A,F)
  \end{align*}
  are the same function, which are bijective due to the proof of the Yoneda lemma (instantiating $F$ with $H_{A'}$).
\end{proof}

By a previous exercise we know the following:
\begin{lemma}
  Let $F\from\cat A\to\cat B$ be a functor and $A\xrightarrow{f} A'$ in $\cat A$.

  Then $f$ is an isomorphism in $\cat A$ if and only if $F(f)$ is an isomorphism in $\cat B$.
\end{lemma}
Thus, we immediately get:
\begin{corollary}[Yoneda Principle]
  Let $\cat A$ be locally small and $A,A'\in\cat A$. Then
  \begin{align*}
    H_A\iso H_{A'} \iff A\iso A'
  \end{align*}
\end{corollary}
Note that the direction from right to left is trivial. Thus, the important consequence of this result is that
\begin{align*}
  H_A(B)=\cat A(B,A)\iso \cat  A(B,A')=H_A'(B) \text{ naturally in } B\in\cat A
\end{align*}
implies $A\iso A'$. Consequently, the following slogan is justified:
\begin{quote}
  \textit{``Two objects are the same if and only if they look the same from all viewpoints.''}
\end{quote}
Note that it is really necessary for the isomorphism to be natural in $B\in\cat A$. As the following example illustrates:
TODO
\section{Consequences of the Yoneda Principle}
Next, we use the insights just gained and we consider (further) consequences of the Yoneda lemma/principle. In particular, we are going to proof the facts mentioned in the introduction.

We start by proving the following ``distributivity''-lemma:

\begin{proposition}
  \label{prop:dis}
  Let $\cat A$ be cartesian closed such that it has coproducts, then for all $A,B,C\in\cat A$
  \begin{align*}
    (A\times B)+(A\times C)\iso A\times (B+C).
  \end{align*}
\end{proposition}

For the proof we need the following lemma, which we are also going to use in the proof that presheaf categories are cartesian closed:
\begin{lemma}
  \label{lem:HApres}
  For any locally small category $\cat A$ and $A\in\cat A$, $H_A$ maps all colimits to limits such that $$H_A(\lim_{\to\mathbf I} D)\iso \lim_{\leftarrow\mathbf I}(H_A\of D)$$ for all small categories $\mathbf I$ and diagrams $D\from \mathbf I\to\cat A$.
\end{lemma}
Now the proof of proposition \ref{prop:dis} goes through smoothly:
\begin{proof}[Proof of proposition \ref{prop:dis}]
  We have 
  \begin{align*}
    \cat A(A\times (B+C),X)&\iso\cat A((B+C)\times A,X)\\
                           &\iso\cat A(B+C,X^A)&\cat A\text{ cartesian closed}\\
                           &\iso\cat A(B,X^A)\times \cat A(C,X^A) &\text{by lemma \ref{lem:HApres}}\\
                           &\iso\cat A(A\times B,X)\times \cat A(A\times C,X) &\cat A\text{ cartesian closed}\\
                           &\iso\cat A((A\times B)+(A\times C),X).&\text{by the dual of lemma \ref{lem:HApres}}
  \end{align*}
  
  All these isomorphisms are natural in $X\in\cat A$. E.g. the second isomorphism is natural in $X\in\cat A$ because it is induced by a transposition. Hence, we have 
  \begin{align*}
    \cat A^\op(X,A\times (B+C))\iso\cat A^\op(X,(A\times B)+(A\times C))
  \end{align*}
 naturally in $X\in\cat A$ and thus we have $(A\times B)+(A\times C)\iso A\times (B+C)$.
\end{proof}

Another rather easy consequence is that left adjoints are unique:

\begin{proposition}
  Let $F,F'\from \cat A\to\cat B$ and $G\from\cat B\to\cat A$ such that $F\dashv G$ and $F'\dashv G$. Then $F\iso F'$
\end{proposition}
\begin{proof}
  By adjointness we have 
  \begin{align*}
    \cat B^\op(B,F(A))\iso\cat B(F(A),B)\iso\cat A(A,G(B))\iso\cat B(F'(A),B)\iso\cat B^\op(B,F'(A))
  \end{align*}
  naturally in $B\in\cat B$. Hence, by what was stated above we have $F(A)\iso F'(A)$ for all $A\in\cat A$. It can be shown that this isomorphsim is natural in $A$ (why?), hence $F\iso F'$.
\end{proof}

\section{Presheaf Category is Cartesian Closed}
In this section we show that for any small category $\cat A$ the presheaf category $[\cat A^\op,\Set]$ is cartesian closed. It is easy to see that it has terminal objects and binary products by defining them ``pointwise''. Showing that exponentials exist is still non-trivial. However the Yoneda lemma provides a straight-forward definition of exponentials and in a special case verifying that this is indeed an exponential is simple. In the general case we need a couple of additional lemmas.

Thus, we are left to show that exponentials exists, i.e. that for functors $G,H\in[\cat A^\op,\Set]$, there is a functor $H^G\in[\cat A^\op,\Set]$ such that
\begin{align*}
  [\cat A^\op,\Set](F,H^G)\iso [\cat A^\op,\Set](F\times G,H)***
\end{align*}
naturally in $F\in[\cat A^\op,\Set]$.

To define $H^G$ on objects $A\in\cat A$ note that by the Yoneda lemma we have
\begin{align*}
  H^G(A)\iso [\cat A^\op,\Set](H_A,H^G)
\end{align*}
and if $H^G$ is the exponential of $G$ and $H$ we certainly have
\begin{align*}
  [\cat A^\op,\Set](H_A,H^G)\iso [\cat A^\op,\Set](H_A\times G,H).
\end{align*}
Clearly this set exists and we can thus define
\begin{align*}
  H^G(A)=[\cat A^\op,\Set](H_A\times G,H)
\end{align*}
on objects.
For maps $A\xrightarrow{f}A'$ we define $H^G(f)=H^G_f\from [\cat A^\op,\Set](H_{A'}\times G,H)\to[\cat A^\op,\Set](H_A\times G,H)$ by
$H^G_f(\alpha)\in[\cat A^\op,\Set](H_A\times G,H)$ for $\alpha\in [\cat A^\op,\Set](H_{A'}\times G,H)$ by
\begin{align*}
  H^G_f(\alpha)_B = (H_g\times 1)\of \alpha_B\from (H_{A}\times G)(B)\to H(B)
\end{align*}
for $B\in\cat A$, which is of course natural.

Therefore, we are left to verify that (***) holds and that it is natural in $F\in[\cat A^\op,\Set]$.
In order to understand the proof better let us first consider the special case where $F=H_A$ is a covariant representable functor for some $A\in\cat A$. Then we immediately have
\begin{align*}
  [\cat A^\op,\Set](H_A,H^G)&\iso H^G(A)&\text{Yoneda lemma}\\
  &=[\cat A^\op,\Set](H_A\times G,H)&\text{definition of }H^G.
\end{align*}

Unfortunately, we have to show that this natural in \emph{all} $F\in[\cat A^\op,\Set]$, not just covariant representable functors, and not every functor is isomorphic to some $H_A$. However, it turns out that each presheaf is a \emph{colimit} of representable functors:

\begin{lemma}
  \label{lem:lim}
  For any small category $\cat A$, any functor $F\from\cat A^\op\to\Set$ is a colimit of representable functors,
  \begin{align*}
    \lim_{\to I\in\mathbf I} H_{\pi(I)}\iso F
  \end{align*}
  for some small category $\mathbf I$ and functor $\pi\from \mathbf I\to \cat A$.
\end{lemma}
\begin{proof}
  First of all, we have to define the shape of the limit, i.e. a small category $\mathbf I$. We take the category of elements of $F$, $F(\cat A)$ with
  \begin{itemize}
  \item objects being pairs $(a,A)$ such that $A\in\cat A$ and $a\in F(A)$;
  \item arrows $f\from(a,A)\to(b,B)$ being arrows $f\from A\to B$ in $\cat A$ such that $F(f)(b)=a$.
  \end{itemize}
  TODO: figure

  and identities and composites defined in the natural way. Clearly, $F(\cat A)$ is small since $\cat A$ is small.

  Next, we have to define a diagram $F(\cat A)\to[\cat A^\op,\Set]$ such that $F$ is the corresponding limit. We simply take $H_-\of\pi$ where $\pi\from F(\cat A)\to\cat A$ is the projection on $\cat A$ defined by 
  \begin{itemize}
  \item $\pi(a,A)=A$ on objects and
  \item $\pi(h)=h$ on arrows in $\cat A$ .
  \end{itemize}

  Now, we have to verify that $F$ is indeed the limit of $H_-\of\pi$ under $F(\cat A)$, i.e. we have to specify a cocone
  \begin{align*}
    (H_{\pi(a,A)}\to F)_{(a,A)\in F(\cat A)}
  \end{align*}
  and show that it is universal amongst all cocones.

  To define $H_{\pi(a,A)}=H_A\to F$ for any $(a,A)\in F(\cat A)$ we recall that $[\cat A^\op,\Set](H_A,F)\iso F(A)$, where the isomorphism from the right to the left is given by $(\widetilde -)$ (as defined in the proof of the Yoneda lemma). Thus, we define the $(a,A)$-component of the cocone to be $\widetilde a$. For this to be a cocone we still have to verify that for each $(b,B)\xrightarrow{f}(a,A)$ (i.e. $B\xrightarrow{f}A$ such that $F(f)(a)=b$)
  \begin{align*}
    \xymatrix{
      H_{\pi(b,B)}=H_B \ar[r]^{\widetilde b} \ar[d]_{H_{\pi(f)}=H_f} & F \\
      H_{\pi(a,A)}=H_A \ar[ru]_{\widetilde a} & 
                         } 
  \end{align*}
  commutes. However, by the naturality statement for $A\in\cat A$ in the Yondea lemma we have that 
  \begin{align*}
    \xymatrix{
    [\cat A^\op,\Set](H_A,F) \ar[r]^{-\of H_f} & [\cat A^\op,\Set](H_B,F) \\
    F(A) \ar[r]^{F(f)} \ar[u]^{(\widetilde-)}& F(B) \ar[u]_{(\widetilde -)}
                         } 
  \end{align*}
  commutes. In particular for any $a\in F(A)$ we have
  \begin{align*}
    \widetilde a\of H_f=\widetilde{F(f)(a)}=\widetilde b
  \end{align*}
  by definition of arrows in $F(\cat A)$.

  Thus, we have shown that this indeed constitues a cocone. Similarly, it can be shown that this cocone is universal, i.e. that $F$ is the limit of $H_-\of\pi$ under $F(\cat A)$.
\end{proof}

Now, are almost in a position where we can prove that presheaf categories are cartesian closed. All we need is the following lemma which states that we can ``pull limits out of products'':
\begin{lemma}
  \label{lem:pull}
  Let $F\from \cat A\to\Set$ be a presheaf. The functor $-\times F\from [\cat A^\op,\Set]\to[\cat A^\op,\Set]$ preserves colomits. That is:
  \begin{align*}
    \lim_{\to \mathbf I} (D\times F)\iso (\lim_{\to \mathbf I} D)\times F
  \end{align*}
  for any small category $\mathbf I$ and functor $D\from\mathbf I\to [\cat A^\op,\Set]$.
\end{lemma}
Finally, we can prove
\begin{proposition}
  For any small category $\cat A$, the category of presheaves $[\cat A^\op,\Set]$ is cartesian closed.
\end{proposition}
\begin{proof}
  It easy to prove that $[\cat A^\op,\Set]$ has terminal objects and binary products (by defining them ``pointwise'').

  With the previous definitions we are left to verify that 
  \begin{align*}
    [\cat A^\op,\Set](F,H^G)\iso [\cat A^\op,\Set](F\times G,H)
  \end{align*}
  naturally in $F\in[\cat A^\op,\Set]$. Using lemma \ref{lem:lim} we have
  \begin{align*}
    F\iso\lim_{\to I\in\mathbf I} H_{\pi(I)}.
  \end{align*}
Hence, we conclude:
  \begin{align*}
    [\cat A^\op,\Set](F,H^G)&\iso [\cat A^\op,\Set](\lim_{\to I\in\mathbf I} H_{\pi(I)},H^G)&\\
                            &\iso \lim_{\leftarrow I\in\mathbf I} [\cat A^\op,\Set](H_{\pi(I)},H^G)&\text{by lemma \ref{lem:HApres}}\\
                            &\iso \lim_{\leftarrow I\in\mathbf I} H^G(\pi(I))&\text{by the Yoneda lemma}\\
                            &\iso \lim_{\leftarrow I\in\mathbf I} [\cat A^\op,\Set](H_{\pi(I)}\times G,H)&\text{by definition of } H^G\\
                            &\iso [\cat A^\op,\Set](\lim_{\to I\in\mathbf I}(H_{\pi(I)}\times G),H)&\text{by lemma \ref{lem:HApres}}\\
                            &\iso [\cat A^\op,\Set](\lim_{\to I\in\mathbf I}(H_{\pi(I)})\times G,H)&\text{by lemma \ref{lem:pull}}\\
                            &\iso[\cat A^\op,\Set](F\times G,H).
  \end{align*}
  All the isomorphisms are natural in $F$. Thus $H^G$ is indeed the exponential of $G$ and $H$.
\end{proof}

\end{document}

% vim mode line:
% vim: ts=2 sts=2 sw=2 expandtab
