\def\pathToRoot{../../}\input{\pathToRoot headers/header_lecturenotes}

\usepackage{todonotes}

\newcommand{\catA}[0]{\cat{A}}
\newcommand{\catB}[0]{\cat{B}}

\newcounter{counter}[section]
\newtheorem{defn}[counter]{Definition}
\newtheorem{exmp}[counter]{Example}
\newtheorem{lemma}[counter]{Lemma}


\title{Natural transformations}
\author{Joachim Bard, Maximilian Wuttke, Nikita Ziuzin}

\begin{document}

\maketitle


\section*{Motivation, definitions and examples}
Morphisms are maps between objects of a category, functors are maps between categories, and finally, natural transformations are maps between functors of the same domain and codomain.

For example, consider a discrete category $\catA$ and an arbitrary category $\catB$.
Now let $\parpair{\catA}{\catB}{F}{G}$ be functors from $\catA$ to $\catB$.
Now because $\catA$ is discrete, we can ignore the morphism-mappings of $F$ and $G$ (since there are only identity morphisms in $\catA$ that are mapped to identity morphisms in $\catB$) and write $F$ and $G$ as families
of objects in $\catB$
$\bigl(F(A)\bigr)_{A\in\catA}$ and $\bigl(G(A)\bigr)_{A\in\catA}$.
When we have a mapping $\alpha$ of $\catB$-morphisms
$\bigl(F(A)\toby{\alpha_A}G(A)\bigr)_{A\in\catA}$
this $\alpha$ could be regarded as a mapping between the functors $F$ and $G$.
Indeed this construction is a special case of the more general definition of natural transformations.

First we define precisely what a natural transformation is in general.
\begin{defn}[Natural transformation]
  \label{def:nat}
  Let $\catA$ and $\catB$ be categories and $\parpair{\catA}{\catB}{F}{G}$ be two functors from $\catA$ to $\catB$.
  Let $\alpha$ be a family of maps
  $\bigl(F(a)\toby{\alpha_A}G(A)\bigr)_{A\in\catA}$.
  Then we define that $\alpha$ is a \demph{natural transformation} from $F$ to $G$ if
  for every objects and maps $A, A' \in \catA, f \from A \to A'$, the square
  \begin{equation}
    \label{eq:nat}
    \begin{array}{c}
      \xymatrix{
        F(A) \ar[r]^{F(f)} \ar[d]_{\alpha_A} & F(A') \ar[d]^{\alpha_{A'}} \\
        G(A) \ar[r]_{G(f)} & G(A')
      }
    \end{array}
  \end{equation}
  commutes.
  The maps $\alpha_A$ are then called the \demph{components} of $\alpha$.
  The commuting square (\ref{eq:nat}) is also called the \demph{naturality axiom} (though this is not an axiom in logical sense but a defined property for some $\alpha$).
\end{defn}

If $\alpha$ is a natural transformation from $F$ to $G$ we can also write this as a diagram:
\[\xymatrix{
  \catA \rtwocell^F_G{\alpha} & \catB
}\]

\begin{exmp}[Discrete categories]
  In the first example, the naturality axiom (\ref{eq:nat}) holds for the mapping
  $\bigl(F(A)\toby{\alpha_A}G(A)\bigr)_{A\in\catA}$,
  because $\catA$ is discrete, so every morphism $f \from A \to A'$ in $\catA$ is an identity morphism (hence $A=A'$). The diagram
  \[
    \xymatrix{
      F(A) \ar[r]^{F(I)} \ar[d]_{\alpha_A} & F(A) \ar[d]^{\alpha_{A}} \\
      G(A) \ar[r]_{G(I)} & G(A)
  }\]
  commutes trivially for every identity morphism $I \from A \to A$. Hence $\alpha$ is a natural transformation according to definition~\ref{def:nat}.
\end{exmp}

\begin{exmp}[Determinants and commutative rings]
  Let $n\in\nat$ and $R$ be a commutative ring.
  We have a forgetful functor $U \from \CRing \to \Mon$ by forgetting the addition of the commutative ring.
  Furthermore the set of all $n \times n$ matrices forms a monoid under matrix multiplication.
  If we map $R \in \CRing$ to $R^{n \times n}$ and ring homomorphisms $f \from R \to S$ to
  $\lambda A \in R^{n \times n}. \bigl( f(A_{i,j}) \bigr)_{1 \le i, j \le n}$,
  i.~e. applying $f$ element-wise to the matrix $A$, we get another functor $M_n \from \CRing \to \Mon$.
  We will now see that the \emph{determinant} is a natural transformation between $M_n$ and $U$.

  First we note that for every ring the determinant of a matrix over this ring is defined canonically.
  Furthermore by definition $\det A \in R$ holds for $A \in R^{n \times n}$.
  We therefore can establish a family of mappings $\det$:
  $\bigl(M_n(R)\toby{\det_R}U(R)\bigr)_{R\in\CRing}$, because $U(R) = R$.
  Now, to show that $\det$ is indeed a natural transformation, we have to show that the following diagram commutes for every ring homomorphism $f \from R \to S$:
  \[\xymatrix{
    M_n(R) \ar[r]^{M_n(f)} \ar[d]_{\det_R} & M_n(S) \ar[d]^{\det_S} \\
    U(R) \ar[r]_{U(f)} & U(s)
  }\]
  By unfolding the definitions of $U$ and $M_n$ and applying functional extensionality, this is equivalent to
  $f(\det_R(f(A))) = \det_S (\bigl( f(A_{i,j}) \bigr)_{1 \le i, j \le n})$
  for every $A \in R^{n \times n}$.
  This already is a standard result of linear algebra.
  Hence $\det$ is really a natural transformation between the functors $M_n$ and $U$. Or, written as a diagram:
  \[\xymatrix{
    \CRing \rtwocell<5>^{M_n}_{U}{\hspace{.7em}\det} & \Mon
  }\]
\end{exmp}

\section*{Functor Category}

Consider a composition $\beta \of \alpha$ between some natural transformations
$\alpha \from F \to G$ and $\beta \from G \to H$.
\[
  \xymatrix@C+1.5em{
    \cat{A}
    \ruppertwocell<8>^F{\alpha}
    \ar[r]|G
    \rlowertwocell<-8>_H{\beta}
    &
    \cat{B}
  }
\]
\vspace{-3ex}\\
We can define such composite natural transformation
\[
  \xymatrix@C+1.5em{
    \cat{A} \rtwocell^F_H{\hspace{.8em}\beta\of\alpha} &\cat{B}
  }
\]
with $(\beta\of\alpha)_A = \beta_A \of \alpha_A$ for all $A \in \cat{A}$.
(Recall that $\beta_A$ and $\alpha_A$ are moprhisms in $\cat{B}$.)

We can also define an identity natural transformation
\[
  \xymatrix@C+1em{
    \cat{A} \rtwocell^F_F{\hspace{.5em}1_F} &\cat{B}
  }
\]
on any functor $F$ with $(1_F)_A = 1_{F(A)}$.

Thus we have a definition of a category:

\begin{defn}[Functor Category]
  The \demph{functor category} from a category $\cat{A}$ to a category
  $\cat{B}$ (written as $\ftrcat{\cat{A}}{\cat{B}}$ or $\cat{B}^\cat{A}$) is a
  category whose objects are the functors from $\cat{A}$ to $\cat{B}$. The maps
  in this category are the natural transformations between functors.
\end{defn}

\begin{exmp}[Functors from a discrete category]
  Consider the discrete category with two objects, $2$. Any functor from $2$ to
  some other category $\cat{B}$ will be a pair of objects in $\cat{B}$. A
  natural transformation between such functors then is a pair of maps between
  these objects.  Note that the functor category $\ftrcat{2}{\cat{B}}$ is
  isomorphic to the product category $\cat{B} \times \cat{B}$ and the
  superscript notation for the functor category $\cat{B}^2$ reflects this
  nicely.
\end{exmp}

\begin{exmp}[Maps in ordered sets]
  Consider ordered sets $A$ and $B$, viewed as categories. The maps
  $\parpairi{A}{B}{f}{g}$ that preserve order can be then considered as
  functors. Note that there is at most one natural transformation between such
  functors and it exists if and only if $f(a) \leq g(a)$ for all $a \in A$.  In
  this case we have the naturality axiom~\eqref{eq:nat} automatically, since
  all diagrams commute in an ordered set. Moreover, the functor category
  $\ftrcat{A}{B}$ is an ordered set itself, where elements are the
  order-preserving maps from $A$ to $B$, and there a morphism $f \leq g$
  between $f$ and $g$ if and only if $f(a) \leq g(a)$ for all $a \in A$.
\end{exmp}

\begin{defn}[Natural isomorphism]
  Let $\cat{A}$ and $\cat{B}$ be categories. A \demph{natural isomorphism}
  between functors from $\cat{A}$ to $\cat{B}$ is an isomorphism in
  $\ftrcat{\cat{A}}{\cat{B}}$.
\end{defn}

This definition can also be given as a lemma:

\begin{lemma}
Let $\xymatrix@1{\cat{A}\rtwocell^F_G{\alpha} &\cat{B}}$ be a natural
transformation.  Then $\alpha$ is a natural isomorphism if and only if
$\alpha_A\from F(A) \to G(A)$ is an isomorphism for all $A \in \cat{A}$.
\end{lemma}

\begin{proof}
  \begin{itemize}
    \item[($\rightarrow$)]
      By definition of natural isomorphism we have a natural transformation
      $\beta\from G \to F$, such that \[\beta \circ \alpha = 1_{F} \text{ and }
      \alpha \circ \beta = 1_{G}. \] We show that $\forall A \in \cat{A}.\,
      \beta_A \circ \alpha_A = 1_{F(A)} \text{ and } \alpha_A \circ \beta_A =
      1_{G(A)}$.  Suppose $A \in \cat{A}$.  We know by definition of the
      identity natural transformation that $(\beta \circ \alpha)_A = 1_{F(A)}$.
      The claim $\beta_A \circ \alpha_A = 1_{F(A)}$ follows by definition of
      composition of natural transformations.  $\alpha_A \circ \beta_A =
      1_{G(A)}$ is shown analogously.
    \item[($\leftarrow$)]
      We know that for all $A \in \cat{A}$ there is a $\beta_A$ such that
      \[\beta_A \circ \alpha_A = 1_{F(A)} \text{ and } \alpha_A \circ \beta_A =
      1_{G(A)}.\] First we prove that $\beta$ is a natural transformation.
      Suppose $f\from A \to A'$.  We know that $\alpha$ is a natural
      transformation:
      \begin{alignat*}{2}
          & & G(f) \circ \alpha_A & = \alpha_{A'} \circ F(f) \\
          &\Rightarrow &\quad G(f) \circ \alpha_A \circ \beta_A & = \alpha_{A'} \circ F(f) \circ \beta_A\\
          &\Rightarrow & G(f) & = \alpha_{A'} \circ F(f) \circ \beta_A\\
          &\Rightarrow & \beta_{A'} \circ G(f) & = \beta_{A'} \circ \alpha_{A'} \circ F(f) \circ \beta_A\\
          &\Rightarrow & \beta_{A'} \circ G(f) & = F(f) \circ \beta_A
      \end{alignat*}
      By similar reasoning to ($\leftarrow$) we get that $\beta$ is the inverse
      of $\alpha$.  Therefore $\alpha$ is a natural isomorphism.
  \end{itemize}
\end{proof}

\begin{defn}[Isomorphic functors]
  We call functors $F$ and $G$ \demph{naturally isomorphic} if there is a
  natural isomorphism from $F$ to $G$, which is just an isomorphism in
  $\ftrcat{\cat{A}}{\cat{B}}$. The notation for isomorphic functors is the same
  as for isomorphic objects: $F \iso G$.
\end{defn}

\begin{defn}[Isomorphic naturally in an object]
  Given functors $\parpair{\cat{A}}{\cat{B}}{F}{G}$, we say that
  \[
  F(A) \iso G(A) \text{ \demph{naturally in} } A
  \]
  if $F$ and $G$ are naturally isomorphic.
\end{defn}

Note that the definition says that we not only have $F(A) \iso G(A)$ for each
individual $A$, but also there is an isomorphism $\alpha_A\from F(A) \to G(A)$
that satisfies the naturality axiom~\eqref{eq:nat}.

\begin{exmp}
  Consider two functors $F, G\from \cat{A} \to \cat{B}$ where $\cat{A}$ is a
  discrete category. We have $F \iso G$ if and only if $F(A) \iso G(A)$ for all
  $A \in \cat{A}$. If it is the case, then we also have that $F(A) \iso G(A)$
  naturally in $A$ for all $A$.

  Be aware that it is not always so: there are examples of categories and
  functors $\parpairi{A}{B}{F}{G}$ where $F(A) \iso G(A)$ for all $A \in
  \cat{A}$, but not naturally in $A$. One of the examples arises if one
  considers two functors that take the objects of the category of finite sets
  and bijections to the sets of permutations and the sets of total orders
  correspondingly.  There one can find isomorphisms between the sets of
  permutations and the sets of total orders, but no natural isomorphism between
  such functors.
\end{exmp}

\section*{Equivalence of Categories}
We now want to examine when two categories are the same.
To that end let us consider the following example.
\begin{exmp}
  \label{exmp:eqv}
  We take two discrete categories $\catA$ and $\catB$ and define two functors $F \from \catA \to \catB$ and $G \from \catB \to \catA$ as depicted below.
  They operate on morphisms in the only possible way.
  \[
    \begin{tikzcd}
      \catA:                                               & & \catB:\\
      A \arrow[loop left] \arrow[dashed, bend left]{rr}{F} & & B \arrow[bend left]{d} \arrow[loop above] \arrow[dashed]{ll}{G}\\
                                                           & & B' \arrow[bend left]{u} \arrow[loop below] \arrow[dashed]{llu}{G}
    \end{tikzcd}
  \]
  The two objects $B$ and $B'$ of $\catB$ are isomorphic.
  By "merging" them we obtain the category $\catA$.
  We now examine how this merging can be described by $F$ and $G$.
  We observe that $G \circ F = 1_{\catA}$.
  But $F \circ G$ is only isomorphic to $1_{\catB}$.
  This isomorphism between the composition of $F$ and $G$ leads to the following definition of equivalence.
\end{exmp}

\begin{defn}
  \label{def:eqv}
  Two categories $\catA$ and $\catB$ are \demph{equivalent} if:
  \begin{itemize}
    \item There are functors $F \from \catA \to \catB$ and $G \from \catB \to \catA$ and
    \item there are natural transformations $\eta \from 1_{\catA} \to G \circ F$ and $\varepsilon \from F \circ G \to 1_\catB$.
  \end{itemize}
  The functors $F$ and $G$ are called \demph{equivalences}.
  We write $\catA \eqv \catB$ if $\catA$ and $\catB$ are equivalent.
\end{defn}

\begin{defn}
  \label{def:surj-on-obj}
  A functor $F \from \catA \to \catB$ is \demph{essentially surjective on objects}
  if for every object $B \in \catB$ there is an object $A \in \catA$ such that $F(A) \iso B$.
\end{defn}

\begin{lemma}
  \label{lem:equiv}
  A functor $F \from \catA \to \catB$ is an equivalence if and only if $F$ is full, faithful and surjective on objects.
\end{lemma}

\begin{exmp}
  Let $K$ be a field.
  We define the category $\catA$ such that objects are of the form $K^n$ for $n \in \NN$ and morphisms are linear functions.
  Moreover we denote the category of finite dimensional vector spaces by $\FDVect_K$.
  Objects in this category are finite dimensional vector spaces and morphisms are linear functions.
  Let $F \from \catA \to \FDVect_K$ be the embedding functor from $\catA$ into $\FDVect_K$, i.e. $F$ maps objects and morphisms to themselves.
  Clearly $F$ is full and faithful.
  For surjectivity on objects we fix an arbitrary object $V \in \FDVect_K$.
  Let $n \in \NN$ be the dimension of $V$.
  From linear algebra we know that $V \iso K^n = F(K^n)$.
  Thus $F$ is essentially surjectivity on objects.
  By lemma~\ref{lem:equiv} we obtain that $\catA \eqv \FDVect_K$.
  %In order to show that $\catA \eqv \FDVect_K$ we have to define a functor $F \from \catA \to \FDVect_K$ which is full, faithful and surjective on objects.
\end{exmp}


\section*{Horizontal Composition}

Given two natural transformations $\alpha$ and $\alpha'$:
\[
\xymatrix@C+.5em{
\cat{A} \rtwocell<4>^F_G{\alpha}   &
\cat{A}' \rtwocell<4>^{F'}_{G'}{\alpha'}   &
\cat{A}'',
}
\]
we can obtain a natural transformation:
\[
\xymatrix@C+.5em{
  \cat{A} \rtwocell<4>^{F' \circ F}_{G' \circ G} &\cat{A}'',
}
\]

This horizontal composition is written $\alpha' * \alpha$.
The following diagram commutes because $\alpha'$ is a natural transformation.
\[
  \begin{tikzcd}
    F'(F(A)) \arrow{r}{F'(\alpha_A)} \arrow{d}{\alpha'_{F(A)}} \arrow{dr}{(\alpha' * \alpha)_A} & F'(G(A)) \arrow{d}{\alpha'_{G(A)}} \\
    G'(F(A)) \arrow{r}{G'(\alpha_A)}                                                            & G'(G(A))
  \end{tikzcd}
\]
So we can define $(\alpha' * \alpha)_A$ either as $G'(\alpha_A) \circ \alpha'_{F(A)}$ or as $\alpha'_{G(A)} \circ F'(\alpha_A)$.
Now we want to show that $\alpha' * \alpha$ is again a natural transformation.
Therefore let $f\from A \to A'$ be a morphism in $\catA$.
\[
  \begin{tikzcd}
    F'(F(A)) \arrow{r}{F'(F(f))} \arrow{d}{\alpha'_{F(A)}} & F'(F(A')) \arrow{d}{\alpha'_{F(A')}} \\
    G'(F(A)) \arrow{r}{G'(F(f))} \arrow{d}{G'(\alpha_A)}   & G'(F(A')) \arrow{d}{G'(\alpha_{A'})}  \\
    G'(G(A)) \arrow{r}{G'(G(f))}                           & G'(G(A'))
  \end{tikzcd}
\]

The upper diagram commutes because $\alpha'$ is a natural transformation.
Since $\alpha$ is a natural transformation we have $G(f) \circ \alpha_A = \alpha_{A'} \circ F(f)$.
By applying $G'$ on both sides we obtain the commutativity of the lower diagram.
The commutativity of both inner diagrams ensures the commutativity of the outer diagram.
This shows that $\alpha' * \alpha$ is indeed a natural transformation.

We can also combine horizontal and vertical composition.
\[
\xymatrix{
\cat{A}
\ruppertwocell<8>^F{\alpha}
\ar[r]|G
\rlowertwocell<-8>_H{\beta} &
\cat{A}'
\ruppertwocell<8>^{F'}{\hspace{.2em}\alpha'}
\ar[r]|{G'}
\rlowertwocell<-8>_{H'}{\hspace{.2em}\beta'} &
\cat{A}''
}
\]

This natural transformations follow the interchange law:
\[
(\beta' \circ \alpha') * (\beta \circ \alpha)
=
(\beta' * \beta) \circ (\alpha' * \alpha)
\from
F' \circ F \to H' \circ H.
\]
Additionally we have $1_{F'} * 1_{F} = 1_{F' \circ F}$, where $1_{F}, 1_{F'}$ and $1_{F' \circ F}$ are the natural transformations which components are the identities on $\cat{A'}, \cat{A'}$ and $\cat{A''}$.

\end{document}

% vim: ts=2 sts=2 sw=2 expandtab
