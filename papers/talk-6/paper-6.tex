\def\pathToRoot{../../}\documentclass[a4paper]{scrbook}
\usepackage[dvipsnames]{xcolor}
\usepackage{nag}
\makeatletter
\@ifclassloaded{beamer}{}
{\usepackage[small,compact]{titlesec}}
\makeatother
\usepackage[utf8]{inputenc}
\usepackage[T1]{fontenc}
\usepackage{lmodern}
\usepackage{color}
\usepackage{parskip}
\usepackage{needspace}
\usepackage{microtype}
\usepackage{mathtools}
\usepackage{xifthen}
\usepackage{xpatch}
\usepackage{enumitem}
\usepackage{mdwlist}
\usepackage{bussproofs}
\EnableBpAbbreviations
\usepackage{tabu}
\usepackage{amssymb}
\usepackage{amsmath}
\usepackage{amsthm}
%grober hack, der den groben hack von parskip bei den amsthm sachen korrigiert
\begingroup
    \makeatletter
       \@for\theoremstyle:=definition,remark,plain\do{%
            \expandafter\g@addto@macro\csname th@\theoremstyle\endcsname{%
                        \addtolength\thm@preskip\parskip
             }%
        }
\endgroup
\usepackage[UKenglish]{babel}
\usepackage{xparse}
\usepackage{adjustbox}
\usepackage{geometry}
\usepackage{booktabs}
\usepackage{multicol}
\usepackage{soul}
\usepackage{calc}
\usepackage{textcase}
\usepackage{stmaryrd}
\usepackage{marvosym}
\usepackage{wasysym}
\usepackage{pifont}
\newcommand{\cmark}{\ding{51}}
\newcommand{\xmark}{\ding{55}}
\usepackage{tikz}
\usetikzlibrary{trees, backgrounds, shapes, chains, decorations.text, decorations.pathreplacing, circuits.logic.IEC, patterns, matrix}
\usepackage{tikz-qtree}
\usepackage{tikzsymbols}
\usepackage{fancyvrb}
\usepackage{fancyhdr}
\usepackage{verbatim}
\usepackage[framemethod=tikz]{mdframed}
\usepackage{lastpage}
\usepackage{pgfpages}
\usepackage{csquotes}
\usepackage{longtable}
\usepackage{ragged2e}
%\usepackage{stackengine}
\usepackage{censor}
\usepackage{expl3}
\usepackage{multirow}
\usepackage{hyperref}
\usepackage{environ}


% Package for Cateogry diagrams:

\usepackage{tikz-cd}



\ifcsdef{labelenumi}{
\renewcommand{\labelenumi}{(\alph{enumi})}
\renewcommand{\labelenumii}{(\roman{enumii})}
}{}

\input{\pathToRoot headers/definitions}



\tikzset{
    normal/.style={draw, semithick},
    n/.style={style=normal, circle, inner sep=1mm, minimum size=8mm},
    l/.style={style=normal, rounded corners=1mm, inner sep=1mm, minimum size=6mm},
    e/.style={style=normal, shorten >=1mm, shorten <=1mm, ->, >=stealth},
    syntax/.style={style=normal, ellipse, minimum height=6mm, minimum width=8mm}, % nodes in syntax trees
    inner/.style={style=normal, minimum size=4mm}, % inner leaves or root in normal trees
    leaf/.style={style=normal, circle, minimum size=4mm}, % leaves in normal trees
    te/.style={style=normal}, % edges in a tree
    be/.style={style=e, dashed} % binding edge
}

\newcommand{\syntaxtree}[1]{ % DEPRECATED - use tikzsyntaxtree
    \begin{tikzpicture}[baseline=(current bounding box.north)]
        \tikzset{grow=down}
        \tikzset{every node/.style={syntax}}
        \tikzset{edge from parent/.style=
            {te,
                edge from parent path={(\tikzparentnode) -- (\tikzchildnode)}}}
        \Tree #1
    \end{tikzpicture}
}

\newenvironment{tikzsyntaxtree}[1][]{
    \begin{tikzpicture}[baseline=(current bounding box.north), #1]
    \tikzset{grow=down}
    \tikzset{every tree node/.style={syntax}}
    \tikzset{edge from parent/.style={te, edge from parent path={(\tikzparentnode) -- (\tikzchildnode)}}}
}{
    \end{tikzpicture}
}


\newcommand{\DisplayScaledProof}{\maxsizebox{\linewidth}{!}{\DisplayProof}}
\newcommand{\DisplayTopProof}{\adjustbox{valign=t}{\DisplayProof}}
\newcommand{\DisplayScaledTopProof}{\adjustbox{valign=t}{\maxsizebox{\linewidth}{!}{\DisplayProof}}}


\newcolumntype{P}[1]{>{\RaggedRight\hspace{0pt}}p{#1}}

\newenvironment{prooftable}
{
    \begin{longtable}{>{\footnotesize}p{0.33\textwidth}>{\footnotesize}p{0.33\textwidth}|>{\footnotesize}P{0.15\textwidth}}
    \normalsize Textbeweis & \normalsize Erklärungen & \normalsize Schlussregel\\\hline
    \endhead
}
{
    \end{longtable}
}


\theoremstyle{definition}
\newtheorem*{definition*}{Definition} % Definition ohne Nummer
\newtheorem*{inferenceRule*}{Schlussregel}



\usepackage{todonotes}

% Shortcuts
\newcommand{\catA}[0]{\cat{A}}
\newcommand{\catB}[0]{\cat{B}}
\newcommand{\fbar}[0]{\bar{f}}
\newcommand{\ftilde}[0]{\tilde{f}}
\newcommand{\gbar}[0]{\bar{g}}
\newcommand{\gtilde}[0]{\tilde{g}}
\newcommand{\dunion}[0]{\sqcup}
\newcommand{\bigdunion}[0]{\bigsqcup}
\renewcommand{\demph}[1]{\emph{#1}} % Definition

% { _ }
\newcommand{\setCon}[1]{\bigl \{ #1 \bigr \}}
% { _ | _ }
\newcommand{\setMap}[2]{\setCon{#1 \,\big|\, #2}}
% ( _ , _ )
\newcommand{\pair}[2]{\left( #1 , #2 \right)}
% [ _ ]
\newcommand{\class}[1]{\left[ #1 \right]}
% < _ >
\newcommand{\choice}[1]{\left< #1 \right>}

% Theorem Environments
\newtheorem{definition}{Definition}
\newtheorem{example}{Example}
\newtheorem{lemma}{Lemma}

% Title page
\title{Special Constructions in Categories}
\author{Maximilian Wuttke}

\begin{document}

\maketitle

\section*{Motivation}
We want to generalize some concepts of sets like the Cartesian product, disjoint union, solution sets and quotient sets.
In this chapter we define six constructions in categories.
We will introduce the constructions \emph{products}, \emph{equalizer} and \emph{pullbacks} and their duals
\emph{coproducts}, \emph{coequalizer} and \emph{pushouts}, respectively.

\section*{Products and Coproducts}

First we want to generalize the Cartesian product to general categories.

\begin{example}[Cartesian Product]
  \label{ex:prod:cart}
  Let $X$ and $Y$ be sets.
  Then we define the \demph{Cartesian Product}
  $$X \times Y := \setMap{\pair{x}{y}}{x \in X, \; y \in Y}. $$
  Also we define the \demph{projection functions} $p_1 \from X \times Y \to X$, $p_2 \from X \times Y \to Y$ in a canonical way:
  \begin{alignat*}{2}
    p_1(x,y) &:= x \\
    p_2(x,y) &:= y
  \end{alignat*}

  Let $A$ be another set and $f_1 \from A \to X, f_2 \from A \to Y$ functions.
  If we want to define a function $\fbar \from A \to X \times Y$, such that
  $p_1(\fbar(a)) = f_1(a)$ and $p_2(\fbar(a)) = f_2(a)$ for every $a \in A$,
  there is only one possible way to define $\fbar \from A \to X \times Y$:
  $$\fbar(a) := \pair{f_1(a)}{f_2(a)}.$$
  We have $p_1(\fbar(a)) = p_1(f_1(a), f_2(b)) = f_1(a)$ and $p_2(\fbar(a)) = p_2(f_1(a), f_2(a)) = f_2(a)$.
  If we have another $\ftilde \from A \to X \times Y$ with $p_1(\ftilde(a)) = f_1(a)$ and $p_2(\ftilde(a)) = f_2(a)$ we show $\fbar(a) = \ftilde(a)$ for every $a \in A$:
  $$\ftilde(a) = \pair{f_1(a)}{f_2(a)} = \pair{p_1(\ftilde(a)}{p_2(\ftilde(a))} = \ftilde(a).$$
  The last step is because we have $\pair{p_1(z)}{p_2(z)} = x$ for every $z \in X \times Y$.
\end{example}

We can use this knowledge to define products in arbitrary categories.

\begin{definition}[Product]
  \label{def:prod}
  Let $\cat{A}$ be a category and $X, Y \in \cat{A}$ objects.
  A \demph{product} of $X$ and $Y$ consists of an object $P$ and maps
  $ \xymatrix{
    X &P \ar[l]_{p_1} \ar[r]^{p_2} & Y
  } $
  with the property that for all objects and maps
  $ \xymatrix{
    X &A \ar[l]_{f_1} \ar[r]^{f_2} & Y
  } $
  in $\cat{A}$, there exists a unique map $\bar{f}\from A \to P$ such that
  \[ \xymatrix{
    & A \ar[ldd]_{f_1} \ar@{.>}[d]|{\bar{f}} \ar[rdd]^{f_2} & \\
    & P \ar[ld]^{p_1} \ar[rd]_{p_2} & \\
    X & & Y
  } \]
  commutes. The maps $p_1$ and $p_2$ are called the \demph{projections}.
\end{definition}

\begin{lemma}[Products are unique up-to isomorphism]
  \label{lem:prod:uniq}
  For every category $\cat{A}$ and every objects $X, Y \in \cat{A}$
  all products of $X$ and $Y$ are isomorphic. Thus we can speak of \emph{the} product of $X$ and $Y$.

  This fact is also true for all the following constructions but we omit the proofs and will do them in the next chapter.
\end{lemma}

\begin{proof}
  Let $X, Y \in \cat{A}$ and $P$ be a product of $X$ and $Y$ together with $p_1 \from P \to X, p_2 \from P \to Y$
  and $P'$ be a product of $X$ and $Y$ together with $p'_1 \from P' \to X, p'_2 \from P' \to Y$.

  Lets reconsider the definition of the \emph{product}:
  For every object $A \in \cat{A}$ and arrows $f_1 \from A \to X, f_2 \from A \to Y$ we get that there exists a unique map
  $\bar{f} \from A \to P$, such that
  \[ \xymatrix{
      &A \ar[ldd]_{f_1} \ar@{.>}[d]|{\bar{f}\vphantom{\bar{\bar{f}}}}
      \ar[rdd]^{f_2}&       \\
              &P \ar[ld]^{p_1} \ar[rd]_{p_2}                  &       \\
      X       &                                               &Y
  } \]
  commutes.
  Now we can instantiate $A$ with $P'$, $f_1$ with $p_1$, $f_2$ with $p_2$. By analogous instantiations we get the commuting diagram
  \[ \xymatrix{
    & P \ar@/_/[ldd]_{p_1} \ar@/^/[rdd]^{p_2} & \\
    & P \ar[ld]^{p_1} \ar@/_/@{.>}[dd]|{\bar{f}} \ar@{.>}[u]|{f} \ar[rd]_{p_2} & \\
    X & & Y \\
    & P' \ar[lu]_{p_1'} \ar@/_/@{.>}[uu]|{\tilde{f}} \ar@{.>}[d]|{f'} \ar[ru]^{p'_2} & \\
    & P' \ar@/^/[luu]^{p'_1} \ar@/_/[ruu]_{p'_2} &
  } \]
  for exactly one $f, f', \tilde{f}, \bar{f}$. With the inner triangles we get:
  \begin{alignat*}{2}
    p_1  \of \tilde{f} &= p'_1 &\qquad p_2  \of \tilde{f} &= p'_2 \\
    p'_1 \of \bar{f}   &= p_1  &       p'_2 \of \bar{f}   &= p_2
  \end{alignat*}
  By substitution we get
  \begin{alignat*}{2}
    p_1   \of (\tilde{f} \of \bar{f}) &= p_1  &\qquad p_2  \of (\tilde{f} \of \bar{f}) &= p_2 \\
    p'_1  \of (\bar{f} \of \tilde{f}) &= p'_1 &       p'_2 \of (\bar{f} \of \tilde{f}) &= p'_2
  \end{alignat*}
  With the outer triangles we get:
  \begin{alignat*}{5}
    p_1  &\;\of&&\; f  &= p_1  &\qquad p_2  &\;\of&\; f  &= p_2 \\
    p'_1 &\;\of&&\; f' &= p'_1 &\qquad p'_2 &\;\of&\; f' &= p'_2
  \end{alignat*}
  Note that $f$ and $f'$ are \emph{unique} and the first equations hold for $f = 1_P$ and $f = \tilde{f} \of \bar{f}$, thus $\tilde{f} \of \bar{f} = 1_P$.
  The second equations holds for $f' = 1_{P'}$ and for $f' = \bar{f} \of \tilde{f}$, thus $\bar{f} \of \tilde{f} = 1_{P'}$.
  Thus $P$ and $P'$ are isomorphic. \qedhere
\end{proof}

\begin{example}[Reversed Cartesian product]
  \label{ex:prod:cart2}
  For sets we can define another product, where we reverse the order of the elements, i.~e.:
  \begin{alignat*}{2}
    X \times' Y &:= \setMap{\pair{y}{x}}{x \in X, \; y \in Y} \\
    p'_1(x,y)   &:= y \\
    p'_2(x,y)   &:= x
  \end{alignat*}
  Now $X \times Y$ and $X \times' Y$ are isomorphic with the involution $i(a, b) := \pair{b}{a}$.
\end{example}

\begin{example}[Posets]
  \label{ex:prod:poset}
  Let $\pair{A}{\le}$ be a poset.
  A \demph{lower bound} for two objects $x, y \in A$ is a element $a \in A$ such that $a \le x$ and $a \le y$.
  A \demph{greatest lover bound} of $x, y \in A$ is a lower bound $z \in A$ of $x$ and $y$ such that for all lower bounds $a \in A$ of $x$ and $y$, $a \le z$.
  With other words: $z$ is a greatest lower bound of $x$ and $y$ if and only if
  $$z \le x \land z \le y \land \forall a \in A, a \le x \land a \le y \Rightarrow a \le z.$$

  When we see the poset $\pair{A}{\le}$ as a category $\catA$ and have a greatest lower bound $z$ of $x, y \in A$, then $z$ is the product of $x$ and $y$.
  To see this, we apply the definition \ref{def:prod}.
  Let $x, y \in \catA$ and $z \in \catA$ be a greatest lower bound of $x$ and $y$.
  Let $a \in \catA$ be another object.
  If we have $f_1 \from a \to x$ and $f_2 \from a \to y$ then by definition of $\catA$, $a$ is a lower bound of $x$ and $y$.
  Now because $z$ is a greatest lower bound, $a \le z$ holds and therefore is a unique map $\fbar \from a \to z$.
  The diagram commutes trivially.
\end{example}

\begin{example}[Direct sum of vector spaces]
  \label{ex:prod:vspace}
  Let $K$ be a field with the operations $+_K$ and $\cdot_K$ and $V, W$ vector spaces over this field with the operations $+_V, \cdot_V, +_W, \cdot_W$.
  Then we define\footnote{One should verify that the defined structure is indeed a vector field.}
  the \demph{direct sum} $V \oplus W$ of $V$ with $V \oplus W := \bigl(V \times W, +, \cdot \bigr)$ and with the following operations:
  \begin{alignat*}{4}
    \pair{v_1}{w_1} + \pair{v_2}{w_2} &:= \pair{v_1 +_V v_2}{w_1 +_W w_2}           &\quad& \forall v_1, v_2 \in V, \forall w_1, w_2 \in W \\
    \alpha \cdot \pair{v}{w}          &:= \pair{\alpha \cdot_V v}{\alpha \cdot_W w} &     & \forall v \in V, \forall w \in W
  \end{alignat*}
  We can now show that $U \oplus V$ is the product of $U$ and $V$ in $\Vect$.
  Therefore we can forget that $U$ and $V$ are vector spaces and follow the proof in example \ref{ex:prod:cart}.
\end{example}

Now we want to abstract the concept of the disjoint union of sets.

\begin{example}[Disjoint union]
  \label{ex:coprod:dunion}
  Let $X$ and $Y$ be sets.
  Let furthermore $0$ and $1$ be some sets with $0 \ne 1$.
  Then we define the \demph{disjoint union} of the sets $X$ and $Y$:
  $$X \dunion Y := (X \times \setCon 0) \cup (Y \times \setCon 1) = \setMap{\pair{x}{0}}{x \in X} \cup \setMap{\pair{y}{1}}{y \in Y}.$$

  Also we define the \demph{coprojection functions} $p_1 \from A \to X \dunion Y$, $p_2 \from Y \to X \dunion Y$:
  \begin{alignat*}{2}
    p_1(x) &:= \pair{x}{0} \\
    p_2(y) &:= \pair{y}{1}
  \end{alignat*}

  Let $A$ be another set and $f_1 \from X \to A, f_2 \from Y \to A$ functions.
  If we want to define a function $\fbar \from X \dunion Y \to A$, such that
  $\fbar(p_1(x)) = f_1(x)$ and $\fbar(p_2(y)) = f_2(y)$ for every $x \in X, y \in Y$,
  there is only one possible way to define $\fbar$:
  \begin{alignat*}{2}
    \fbar(x, 0) &:= f_1(x) \\
    \fbar(y, 1) &:= f_2(y).
  \end{alignat*}
\end{example}

We can again use this knowledge to define coproducts in arbitrary categories.

\begin{definition}[Coproduct]
  \label{def:coproduct}
  Let $\cat{A}$ be a category and $X, Y \in \cat{A}$.
  A \demph{coproduct} of $X$ and $Y$ consists of an object $S$ and maps
  $ \xymatrix{
    X \ar[r]^{p_1} & S & Y \ar[l]_{p_2}
  } $
  with the property that for all objects and maps
  $ \xymatrix{
    X \ar[r]^{f_1} & A & Y \ar[l]_{f_2}
  } $
  in $\cat{A}$, there exists a unique map $\bar{f}\from S \to A$ such that
  \[ \xymatrix{
    X \ar[rdd]_{f_1}\ar[rd]^{p_1} & & Y \ar[ldd]^{f_2} \ar[ld]_{p_2} \\
    & S \ar@{.>}[d]|{\bar{f}} & \\
    & A &
  } \]
  commutes. The maps $p_1$ and $p_2$ are called the \demph{coprojections}.
\end{definition}

\begin{example}[Direct sum of vector spaces]
  \label{ex:coprod:vspace}
  In the example \ref{ex:prod:vspace} we have seen that the direct sum of two vectorspaces $U$ and $V$ is the product of $V$ and $W$ in $\Vect$.
  We will now see that it is \emph{also} the coproduct.
  In general products and coproducts are not the same as we have seen in the examples \ref{ex:prod:cart} and \ref{ex:coprod:dunion}.
  First we have to define the two coprojection functions $p_1 \from V \to V \oplus W, p_2 \from W \to V \oplus W$:
  \begin{alignat*}{2}
    p_1(x) &:= \pair{x}{0} \\
    p_2(y) &:= \pair{0}{y}
  \end{alignat*}
  Then let $A \in \Vect$ and $f_1 \from V \to A, f_2 \from W \to A$ be vector morphisms.
  We define the vector morphism $\fbar \from V \oplus W \to A$ with $\fbar(v, w) := f_1(v) + f_2(w)$.
  By construction and because $f_1$ and $f_2$ are vector morphims we have:
  \begin{alignat*}{4}
    \fbar(p_1(u)) &= \fbar(u, 0) &= f_1(u) + f_2(0) &= f_1(u) \\
    \fbar(p_2(v)) &= \fbar(0, v) &= f_1(0) + f_2(v) &= f_2(v).
  \end{alignat*}
  Now we have to show that $\fbar$ is the only vector morphims with this property.
  Let $\ftilde \from V \oplus W \to A$ be another vector morphims with $\ftilde(p_1(u)) = f_1(u)$ and $\ftilde(p_2(v)) = f_2(v)$.
  Then we have for every $v \in V$ and $w \in W$:
  $$ \fbar(v,w) = f_1(v) + f_2(w)\ = \ftilde(p_1(v)) + \ftilde(p_2(w)) = \ftilde(p_1(v) + p_2(w)) = \ftilde(v, w). $$
  Thus $\fbar = \ftilde$ and $V \oplus W$ is indeed the product of $V$ and $W$ in the the category $\Vect$.
\end{example}

\section*{Equalizers and Coequalizer}

%% XXX Weiterhin erst Beispiel und dann Definition oder erst Definition und dann Beispiel?
Now We present abstract concepts and give examples.

\begin{definition}[Equalizer]
  \label{def:equa}
  Let $\cat{A}$ be a category and
  $ \xymatrix{
    X \ar@<.5ex>[r]^s \ar@<-.5ex>[r]_t & Y
  } $
  a diagram in $\cat{A}$.
  An \demph{equalizer} of this diagram is a commuting diagram called \demph{fork} \\
  $ \xymatrix{
    E \ar[r]^i & X \ar@<.5ex>[r]^s \ar@<-.5ex>[r]_t & Y
  } $
  with the property that for all other forks
  $ \xymatrix{
    A \ar[r]^f & X \ar@<.5ex>[r]^s \ar@<-.5ex>[r]_t & Y
  } $
  in $\cat{A}$, there exists a unique map $\bar{f}\from A \to E$ such that
  \[ \xymatrix{
    A \ar[rd]^f \ar@{.>}[dd]|{\bar{f}} & & \\
    & X \ar@<.5ex>[r]^s \ar@<-.5ex>[r]_t & Y \\
    E \ar[ru]_i & &
  } \]
  commutes.
\end{definition}

\begin{example}[Solution sets]
  Let $X, Y$ be sets and $s, t \from X \to Y$ functions.
  We define $E := \setMap{x \in X}{s(x)=t(x)}$.
  Because $E \subseteq X$ we can define the \emph{inclusion function} $i \from E \to X$ with $i(e) := e$.
  Now we show that the fork
  $ \xymatrix{
    E \ar[r]^i & X \ar@<.5ex>[r]^s \ar@<-.5ex>[r]_t & Y
  } $
  is the equalizer of $s$ and $t$. This diagram commutes by construction.
  Let $A$ be another set and
  $ \xymatrix{
    A \ar[r]^f & X \ar@<.5ex>[r]^s \ar@<-.5ex>[r]_t & Y
  } $
  be another fork.
  We then define $\fbar \from A \to E$ with $\fbar(a) = f(a)$.
  Then $i \of \fbar = f$ holds by definition.
  Let $\ftilde \from A \to E$ be another function such that $i \of \ftilde = f$.
  Then $\ftilde(a) = i (\ftilde(a)) = f(a) = i (\fbar(a)) = \fbar(a)$.
\end{example}

\todo{Mehr Beispiele dazu?}

\begin{definition}[Coequalizer]
  \label{def:coequa}
  Let $\cat{A}$ be a category and
  $ \xymatrix{
    X \ar@<.5ex>[r]^s \ar@<-.5ex>[r]_t & Y
  } $
  a diagram in $\cat{A}$.
  A \demph{coequalizer} of this diagram is a commuting diagram called \demph{cofork} \\
  $ \xymatrix{
    X \ar@<.5ex>[r]^s \ar@<-.5ex>[r]_t & Y \ar[r]^i & C
  } $
  with the property that for all other coforks
  $ \xymatrix{
    X \ar@<.5ex>[r]^s \ar@<-.5ex>[r]_t & Y \ar[r]^f & A
  } $
  in $\cat{A}$, there exists a unique map $\bar{f}\from C \to A$ such that
  \[ \xymatrix{
    & & A  \\
    X \ar@<.5ex>[r]^s \ar@<-.5ex>[r]_t & Y \ar[ru]^f \ar[rd]_i \\
    & & C \ar@{.>}[uu]|{\bar{f}}
  } \]
  commutes.
\end{definition}

For our next example we need the definition of the reflexive transitive closure.

\begin{definition}[Symmetric transitive closure]
  Let $R$ be a binary relation over a set $X$.
  $\hat{R} := R \cup R^{-1}$ is the symmetric closure of $R$ where $R^{-1}$ is defined by
  $R^{-1} := \setMap{\pair{b}{a}}{\pair{a}{b} \in R}$.
  The \demph{symmetric transitive closure} of $R$ is $\sim := \bigcup_{n\in\nat} {(\hat{R})^n}$
  where ${\hat R}^n$ is defined recursively with
  ${\hat R}^0 := \setMap{\pair{a}{a}}{a \in X}$ and
  ${\hat R}^{n+1} := R_n \of \hat R = \setMap{\pair{a}{c}}{\pair{a}{b} \in {\hat R}^n \land \pair{b}{c} \in \hat R}$.
  It can be shown that $\sim$ is the \emph{least equivalence relation} containing $R$.
\end{definition}

\begin{example}[Quotient sets]
  Let $X, Y$ be sets and $s, t \from X \to Y$ be functions.
  We have to give a set $C$ and a function $i$ such that the diagram
  $ \xymatrix{
    X \ar@<.5ex>[r]^s \ar@<-.5ex>[r]_t & Y \ar[r]^i & C
  } $
  commutes.
  Therefore we define the relation
  $R := \setMap{\pair{s(a)}{t(a)}}{a \in Y} \subseteq Y^2$ the relation of values in $Y$ that correspond from the same $x \in X$.
  Let $\sim \subseteq Y^2$ be the symmetric transitive closure of $R$.
  Now we define $C := Y / \sim = \setMap{\class y}{y \in Y}$ where $\class y := \setMap{y'}{y \sim y'}$ are the equivalence classes.
  For the function $i \from Y \to E$ we choose $i(y) := \class y$.
  Then the diagram
  $ \xymatrix{
    X \ar@<.5ex>[r]^s \ar@<-.5ex>[r]_t & Y \ar[r]^i & C
  } $
  commutes because for $x \in X$, $s(x)$ and $t(x)$ are in $R$ and therefore also in $\sim$; because $s(x) \sim t(x)$ we have $i(s(x)) = \class{s(x)} = \class{t(x)} = i(t(x))$.
  Now let
  $ \xymatrix{
    X \ar@<.5ex>[r]^s \ar@<-.5ex>[r]_t & Y \ar[r]^f & A
  } $
  be a cofork.
  By induction we can show that $f$ ``does not care'' about equal values, i.~e. $f(y_1) = f(y_2)$ whenever $y_1 \sim y_2$.

  To define the function $\fbar \from C \to A$ we need a choice function $\choice \cdot$ that returns for every equivalence class in $\sim$ a member of it, i.~e.
  $\forall c \in C, \forall y \in e, \choice{e} \sim y$.
  Therefore we also know $\forall c \in C, \class {\choice c} = c$.

  Now we can define $\fbar \from C \to A$ by $\fbar(e) := f(\choice e)$.
  We now have  $\fbar(i(y)) = f(\choice{\class y}) = f(y)$ because $\choice{\class y} \sim y$ (by reflexivity of $\sim$).
  Now we only have to show that this $\fbar$ is unique. So we assume another function $\ftilde \from C \to A$ with $\ftilde \of i = f$. But then we have
  $$ \fbar(c) = f(\choice c) = \ftilde(i(\choice c)) = \ftilde(\class{\choice c}) = \ftilde(c).$$
\end{example}


\section*{Pullbacks and Pushouts}

\begin{definition}[Pullback]
  \label{def:pullback}
  Let $\cat{A}$ be a category and
  $ \xymatrix{
    X \ar[r]^s & Z & Y\ar[l]_t \\
  } $
  a diagram in $\cat{A}$.
  A \demph{pullback} of this diagram is a \demph{pullback square}
  \[ \xymatrix{
    P \ar[r]_{p_2} \ar[d]^{p_1} & Y \ar[d]_{t} \\
    X \ar[r]^s & Z
  } \]
  with the property that for all other pull back squares
  \[ \xymatrix{
    A \ar[r]_{f_2} \ar[d]^{f_1} & Y \ar[d]_{t} \\
    X \ar[r]^s & Z
  } \]
  in $\cat{A}$, there exists a unique map $\bar{f}\from A \to P$ such that
  \[ \xymatrix{
    A \ar[drr]^{f_2} \ar[ddr]_{f_1} \ar@{.>}[dr]|{\bar{f}} & & \\
    & P \ar[r]_{p_2} \ar[d]^{p_1} & Y \ar[d]_{t} \\
    & X \ar[r]^s & Z
  } \]
  commutes.
\end{definition}

The pullback is also called ``fibred product''. This will be clear in the following example:
\todo{Besser erklären, was das Beispiel damit zu tun hat}

\begin{example}[Fibred product]
  When the pullback square
  \[ \xymatrix{
    P \ar[r]_{p_2} \ar[d]^{p_1} & Y \ar[d]_{t} \\
    X \ar[r]^s & Z
  } \]
  is the pullback of
  $ \xymatrix{
    X \ar[r]^s & Z & Y\ar[l]_t \\
  } $
  and $Z$ is a terminal object then $P$ is the product of $X$ and $Y$.
\end{example}

\begin{example}[Pullbacks in sets]
  Let $X, Y$ be sets and $s \from X \to Z, t \from Y \to Z$ be functions.
  Then $P := \setMap{\pair{x}{y}}{s(x) = t(y)}$ with the projection functions
  $p_1(x,y) := x$ and $p_2(x, y) = y$ is the pullback of
  $ \xymatrix{
    X \ar[r]^s & Z & Y\ar[l]_t \\
  } $.
  The pullback square commutes by construction.
  Now let $A$ be a set and $f_1 \from A \to X, f_2 \from A \to Y$ be functions with $s \of f_1 = t \of f_2$.
  Then we define $\fbar \from A \to P$ by
  $\fbar(a) := \pair{f_1(a)}{f_2(a)}$.
  This is well defined because for $a \in A$ we have $\fbar(a) = \pair{f_1(a)}{f_2(a)} \in P \Leftrightarrow s(f_1(a) = t(f_2(a))$ wich is true by assumption.
  We also have $p_1(\fbar(a)) = f_1(a)$ and $p_2(\fbar(a)) = f_2(a)$ by constructions.
  We also see that $\fbar$ is unique with this property.
\end{example}

\begin{definition}[Pushout]
  \label{def:pushout}
  Let $\cat{A}$ be a category and
  $ \xymatrix{
    X & Z \ar[l]_s \ar[r]^t & Y \\
  } $
  a diagram in $\cat{A}$.
  A \demph{pushout} of this diagram is a \demph{pushout square} \\
  \[ \xymatrix{
    Z \ar[r]_s \ar[d]^t & X \ar[d]_{p_1} \\
    Y \ar[r]^{p_2} & P
  } \]
  with the property that for all other pushout squares
  \[ \xymatrix{
    Z \ar[r]_s \ar[d]^t & X \ar[d]_{f_1} \\
    Y \ar[r]^{f_2} & A
  } \]
  in $\cat{A}$, there exists a unique map $\bar{f}\from P \to A$ such that
  \[ \xymatrix{
    Z \ar[r]_s \ar[d]^t & X \ar[d]_{p_1} \ar[ddr]^{f_1} & \\
    Y \ar[r]^{p_2} \ar[drr]_{f_2}& P \ar@{.>}[dr]|{\bar{f}} & \\
    & & A  \\
  } \]
  commutes.
\end{definition}

\todo{Beispiele}

\section*{Conclusion}
We have seen much analogous definitions and proofs.
Our definitions follow a simple ``pattern'':
\begin{itemize}
  \item First we have a some objects or a commuting diagram ($D_0$).
  \item Then we ``insert'' a object $P$ and get a new commuting diagram ($D_1$) where we ``connect'' $P$ with the objects of $D_0$ with ``projection functions''.
  \item Then we ``replace'' this object with a new object, called $A$, and replace the ``projection functions'' with new functions. We get a new diagram ($D_2$).
  \item For each $A$ there exists a unique map $\fbar$ such if we ``merge'' $D_1$ and $D_2$ we get a new commuting diagram $D_3$, where $A$ and $P$ are connected with $\fbar$.
\end{itemize}

In the next chapter we will see that all this six constructions are instances of a more general concept: limits and colimits.
We will formalize the ``pattern'' above.

We have assumed that all instances of the six special constructions above are unique up-to isomorphism
and have shown this fact only for products. We will see a proof for all limits and colimits.

\end{document}

% vim: ts=2 ss=2 sw=2 expandtab
