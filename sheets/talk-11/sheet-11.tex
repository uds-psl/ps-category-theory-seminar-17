\def\pathToRoot{../../}\documentclass{article}

\usepackage{nag}
\makeatletter
\@ifclassloaded{beamer}{}
{\usepackage[small,compact]{titlesec}}
\makeatother
\usepackage[utf8]{inputenc}
\usepackage[T1]{fontenc}
\usepackage{lmodern}
\usepackage{color}
\usepackage{parskip}
\usepackage{needspace}
\usepackage{microtype}
\usepackage{mathtools}
\usepackage{xifthen}
\usepackage{xpatch}
\usepackage{enumitem}
\usepackage{mdwlist}
\usepackage{bussproofs}
\EnableBpAbbreviations
\usepackage{tabu}
\usepackage{amssymb}
\usepackage{amsmath}
\usepackage{amsthm}
%grober hack, der den groben hack von parskip bei den amsthm sachen korrigiert
\begingroup
    \makeatletter
       \@for\theoremstyle:=definition,remark,plain\do{%
            \expandafter\g@addto@macro\csname th@\theoremstyle\endcsname{%
                        \addtolength\thm@preskip\parskip
             }%
        }
\endgroup
\usepackage[UKenglish]{babel}
\usepackage{xparse}
\usepackage{adjustbox}
\usepackage{geometry}
\usepackage{booktabs}
\usepackage{multicol}
\usepackage{soul}
\usepackage{calc}
\usepackage{textcase}
\usepackage{stmaryrd}
\usepackage{marvosym}
\usepackage{wasysym}
\usepackage{pifont}
\newcommand{\cmark}{\ding{51}}
\newcommand{\xmark}{\ding{55}}
\usepackage{tikz}
\usetikzlibrary{trees, backgrounds, shapes, chains, decorations.text, decorations.pathreplacing, circuits.logic.IEC, patterns, matrix}
\usepackage{tikz-qtree}
\usepackage{tikzsymbols}
\usepackage{fancyvrb}
\usepackage{fancyhdr}
\usepackage{verbatim}
\usepackage[framemethod=tikz]{mdframed}
\usepackage{lastpage}
\usepackage{pgfpages}
\usepackage{csquotes}
\usepackage{longtable}
\usepackage{ragged2e}
%\usepackage{stackengine}
\usepackage{censor}
\usepackage{expl3}
\usepackage{multirow}
\usepackage{hyperref}
\usepackage{environ}


% Package for Cateogry diagrams:

\usepackage{tikz-cd}



\ifcsdef{labelenumi}{
\renewcommand{\labelenumi}{(\alph{enumi})}
\renewcommand{\labelenumii}{(\roman{enumii})}
}{}

\input{\pathToRoot headers/definitions}



\tikzset{
    normal/.style={draw, semithick},
    n/.style={style=normal, circle, inner sep=1mm, minimum size=8mm},
    l/.style={style=normal, rounded corners=1mm, inner sep=1mm, minimum size=6mm},
    e/.style={style=normal, shorten >=1mm, shorten <=1mm, ->, >=stealth},
    syntax/.style={style=normal, ellipse, minimum height=6mm, minimum width=8mm}, % nodes in syntax trees
    inner/.style={style=normal, minimum size=4mm}, % inner leaves or root in normal trees
    leaf/.style={style=normal, circle, minimum size=4mm}, % leaves in normal trees
    te/.style={style=normal}, % edges in a tree
    be/.style={style=e, dashed} % binding edge
}

\newcommand{\syntaxtree}[1]{ % DEPRECATED - use tikzsyntaxtree
    \begin{tikzpicture}[baseline=(current bounding box.north)]
        \tikzset{grow=down}
        \tikzset{every node/.style={syntax}}
        \tikzset{edge from parent/.style=
            {te,
                edge from parent path={(\tikzparentnode) -- (\tikzchildnode)}}}
        \Tree #1
    \end{tikzpicture}
}

\newenvironment{tikzsyntaxtree}[1][]{
    \begin{tikzpicture}[baseline=(current bounding box.north), #1]
    \tikzset{grow=down}
    \tikzset{every tree node/.style={syntax}}
    \tikzset{edge from parent/.style={te, edge from parent path={(\tikzparentnode) -- (\tikzchildnode)}}}
}{
    \end{tikzpicture}
}


\newcommand{\DisplayScaledProof}{\maxsizebox{\linewidth}{!}{\DisplayProof}}
\newcommand{\DisplayTopProof}{\adjustbox{valign=t}{\DisplayProof}}
\newcommand{\DisplayScaledTopProof}{\adjustbox{valign=t}{\maxsizebox{\linewidth}{!}{\DisplayProof}}}


\newcolumntype{P}[1]{>{\RaggedRight\hspace{0pt}}p{#1}}

\newenvironment{prooftable}
{
    \begin{longtable}{>{\footnotesize}p{0.33\textwidth}>{\footnotesize}p{0.33\textwidth}|>{\footnotesize}P{0.15\textwidth}}
    \normalsize Textbeweis & \normalsize Erklärungen & \normalsize Schlussregel\\\hline
    \endhead
}
{
    \end{longtable}
}


\theoremstyle{definition}
\newtheorem*{definition*}{Definition} % Definition ohne Nummer
\newtheorem*{inferenceRule*}{Schlussregel}

\usepackage{titling}
\geometry{a4paper,left=2cm,right=2cm,top=2cm,bottom=3cm}


\newcommand{\licenseccjuliachristian}{\def\islicenseccjuliachristian{}}
\newcommand{\suppresslicense}{\def\issuppresslicense{}}


\AtBeginDocument{
    \pagestyle{fancy}
    \renewcommand{\headrulewidth}{0pt}
    \renewcommand{\footrulewidth}{1pt}
    \fancyhead{}
    \fancyfoot[C]{\thepage~/~\pageref{LastPage}}
    \fancyfoot[R]{\footnotesize exercise sheet from \\ \theauthor}

}


\newcommand{\pgbreakhere}{\Needspace*{4\baselineskip}}
\newcommand{\pgbreakHere}{\Needspace*{10\baselineskip}}
\newcommand{\pgbreakHERE}{\Needspace*{15\baselineskip}}

\newcommand{\raisedrule}[2][0em]{\leavevmode\leaders\hbox{\rule[#1]{1pt}{#2}}\hfill\kern0pt}

% inspired by http://tex.stackexchange.com/questions/242294/suppress-parskip-only-after-a-specific-paragraph
\makeatletter
\newlength\noparskip@parskip % used to store a backup of the parskip value
\newboolean{noparskip@triggered} % flag to indicate that noparskip was run in the current paragraph
\setboolean{noparskip@triggered}{false}
\newboolean{noparskip@active} % flag to indicate that parskip should be restored after this paragraph
\setboolean{noparskip@active}{false}
\let\noparskip@par\par % store a backup of the \par command
\@setpar{% redefine \par with the means of ltpar.dtx to stay compatible to enumerate and itemize
    \ifhmode% since we're counting occurrences of \par, \par\par would be a problem, so check that we are actually ending a paragraph
        \ifthenelse{\boolean{noparskip@active}}{%
            \setlength\parskip\noparskip@parskip% restore parskip
            \setboolean{noparskip@active}{false}% remember not the restore parskip again
        }{}%
        \ifthenelse{\boolean{noparskip@triggered}}{%
            \ifthenelse{\boolean{noparskip@active}}{}{
                % we are triggering noparskip and not currently in a noparskip already
                \setlength\noparskip@parskip\parskip % copy the current parskip into the backup variable
            }%
            \setboolean{noparskip@triggered}{false}% paragraph is ending, so noparskip is no longer triggered
            \setlength\parskip{0pt}% no parskip when the next paragraph begins
            \setboolean{noparskip@active}{true}% parskip must be restored by the next par
        }{}%
    \fi%
    \noparskip@par% run the original par command
}
\def\noparskip@backout{%
    \ifthenelse{\boolean{noparskip@active}}{%
        % a list is beginning and parskip is currently set to zero, wich would mess up the list
        \setlength\parskip{\noparskip@parskip}% restore parskip before the list begins
        \setboolean{noparskip@active}{false}%
    }{}%
    \setboolean{noparskip@triggered}{false}% there's no sense in keeping noparskip triggered throughout a list
}
\xpretocmd\begin{%
    \ifstrequal{#1}{enumerate}{\noparskip@backout}{}%
    \ifstrequal{#1}{itemize}{\noparskip@backout}{}%
    \ifstrequal{#1}{list}{\noparskip@backout}{}%
    \ifstrequal{#1}{proof}{\noparskip@backout}{}%
}{}{}
\def\noparskip{%
    \leavevmode% ensure that we are within a paragraph
    \setboolean{noparskip@triggered}{true}% trigger noparskip
}
\makeatother

\newcommand{\noparskipworkaround}{} % DEPRECATED and no longer needed


\newcommand{\head}[1]{
    {
        \setlength{\parskip}{0pt}
        \hrule height 1pt
        \vspace{.2cm}
        Saarland University \hfill Category Theory Seminar 2017\par
        Programming Systems Lab \hfill \small\url{https://courses.ps.uni-saarland.de/ct_ss17/}\par
        \tiny\raisedrule[0mm]{1pt}
        \vspace{2ex}
        \begin{center}
            \Large
            \textbf{#1}\par
            \raisedrule[2mm]{1pt}
        \end{center}
        \vspace{3ex}
    }
}

\newenvironment{leftframedparagraph}{\begin{mdframed}[hidealllines = true, leftline = true, innerleftmargin = 2ex, innerrightmargin = 0pt,
innertopmargin = 0pt, innerbottommargin = 2pt, skipabove=2ex, skipbelow=1ex, outerlinewidth = 0ex, innerlinewidth = 0.5ex]}{\end{mdframed}}
\newenvironment{leftframed}{\begin{mdframed}[hidealllines = true, leftline = true, innerleftmargin = 2ex, innerrightmargin = 0pt,
innertopmargin = 0pt, innerbottommargin = 0pt, skipabove=2ex, skipbelow=1ex, outerlinewidth = 0ex, innerlinewidth = 0.5ex]}{\end{mdframed}}

%%% Local Variables:
%%% mode: latex
%%% TeX-master: t
%%% End:


\newcommand{\uebunghead}[3][Exercise sheet:]{\def\sheetid{#2}\head{#1 #2\ifthenelse{\isundefined{\issolution}}{}{ \ifthenelse{\isundefined{\ismarking}}{(Possible solutions)}{(Marking)}} \\ #3}}

\licenseccjuliachristian


\newcommand{\amountofpoints}[1]{\ifstrequal{#1}{1}{1~Punkt}{#1~Punkte}}


% marking implies solution
\ifthenelse{\isundefined{\ismarking}}{}{\def\issolution{}}


%%%Environments
\newcounter{ExamExerciseCounter} % will only be used in exams, but must be defined here so ExerciseCounter can be reset when ExamExericise counts
\setcounter{ExamExerciseCounter}{0}
\newcounter{ExerciseCounter}[ExamExerciseCounter]
\setcounter{ExerciseCounter}{0}

\newcommand{\ExerciseNumber}{\sheetid.\arabic{ExerciseCounter}}
\renewcommand{\theExerciseCounter}{\ExerciseNumber}

\newcommand{\ExercisePointHook}[1]{}

%Aufgaben-Umgebung
\NewDocumentEnvironment{exercise}{od<>}{
    \refstepcounter{ExerciseCounter}
    \pgbreakhere
    \vspace{1ex}\textbf{Exercise\ \ExerciseNumber}%
    \IfNoValueF{#1}{ \emph{(#1)}}%
    \IfNoValueF{#2}{\hfill(\amountofpoints{#2})}%
    \IfNoValueF{#2}{\ExercisePointHook{#2}}%
    \noparskip\par\nopagebreak
}{
    \par
    \vspace{2ex}
}

\newcommand{\exercisesOnly}[1]{\ifthenelse{\isundefined{\issolution}}{#1}{}}

%Loesungs-Umgebung
\newenvironment{answer}
{
    \ifthenelse{\isundefined{\issolution}}
    {
        \comment
    }{
        \vspace{1ex}\textsl{Sample solution \ExerciseNumber}\noparskip\par\nopagebreak
    }
}{
    \ifthenelse{\isundefined{\issolution}}
    {
    }{
        \vspace{1ex}
        \hspace*{\fill}
    }
}

\newenvironment{marking}
{%
    \ifthenelse{\isundefined{\ismarking}}%
    {%
        \comment%
    }{%
        \color{red}
    }%
}{%
    \ifthenelse{\isundefined{\ismarking}}%
    {%
    }{%
    }%
}

\newenvironment{example}{\begin{leftframedparagraph}\paragraph{Example:}}{\end{leftframedparagraph}}
\newenvironment{hint}{\paragraph{Hint:}}{}
\newenvironment{caution}{\paragraph{Caution:}}{}
\newenvironment{definition}[1]{\begin{leftframedparagraph}\paragraph{Definition (#1):}}{\end{leftframedparagraph}}

\usepackage{semantic}

\begin{document}

% Use Basis x or Talk x, where x is the number of the session
\uebunghead{Talk 11}{Dependent Types}
\author{Nikita Ziuzin}

\begin{hint}
  Read sections $2.1.3, 2.1.5, 2.1.6$ in Martin Hofmann's \href{http://www.irif.fr/~mellies/mpri/mpri-ens/articles/hofmann-syntax-and-semantics-of-dependent-types.pdf}{Syntax and Semantics of Dependent Types} and section $1$ in Simon Huber's \href{http://www.cse.chalmers.se/~simonhu/misc/lic.pdf}{A Model of Type Theory in Cubical Sets}.
\end{hint}

\begin{exercise}
  Give the sets $\mathit{Ty}_F(\Gamma)$ and $\mathit{Ter}_F(\Gamma, A)$ for
  the presheaf model.
\end{exercise}
\begin{answer}
  \[
    \mathit{Ty}_F(\Gamma) := \mathit{Psh(\int_\cat{C} \Gamma)}
  \]
  \[
    \mathit{Ter}_F(\Gamma, A) := \{(a\rho \in A\rho)_{I \in \cat{C}, \rho \in \Gamma(I)} | \forall f \from J \to I, a\rho f = a(\rho f) \}
  \]
\end{answer}

\begin{exercise}
  Let $(\cat C, \mathcal F)$ be a category with families.  Check that the
  following square is a pullback in $\cat{C}$:
  \[
    \begin{tikzcd}
      \Delta.A\sigma \arrow{r}{(\sigma \circ p_\Delta, q)} \arrow{d}{p_\Delta} & \Gamma.A \arrow{d}{p_\Gamma}\\
      \Delta \arrow{r}{\sigma} & \Gamma
    \end{tikzcd}
  \]
  where $\Delta.A \vdash q : A p_\Delta$
\end{exercise}
\begin{answer}
  Let $\Delta'$ be a context and $p' \from \Delta' \to \Delta$ and $\sigma'
  \from \Delta' \to \Gamma.A$ be two morphisms such that $\sigma \circ p' =
  p_\Gamma \circ \sigma'$. Therefore the diagram
  \[
    \begin{tikzcd}
      \Delta' \arrow[bend right=30]{ddr}{p'} \arrow[dashed]{dr}{\delta} \arrow[bend left=30]{drr}{\sigma'}&   &\\
          & \Delta.A\sigma \arrow{r}{(\sigma \circ p_\Delta, q)} \arrow{d}{p_\Delta} & \Gamma.A \arrow{d}{p_\Gamma}\\
          & \Delta \arrow{r}{\sigma} & \Gamma
    \end{tikzcd}
  \]
  (without $\delta$) is commutative. We must show that there exists a unique
  $\delta \from \Delta' \to \Delta.A\sigma$ such that $(\sigma \circ p_\Delta,
  q) \circ \delta = \sigma'$ and $p_\Delta \circ \delta = p'$.

  We can decompose $\sigma'$ as $\sigma' = 1 \circ \sigma' = (p_\Gamma,
  q_\Gamma) \circ \sigma' = (p_\Gamma \circ \sigma', q_\Gamma\sigma') = (\sigma
  \circ p', M)$, where $M \equiv q_\Gamma\sigma'$ (of type $\Delta' \vdash
  A(\sigma \circ p') ~\text{type}$). Let $\delta$ be the morphism
  \[
    \delta \equiv (p', M) \from \Delta' \to \Delta.A\sigma
  \]
  which is suitable since $p_\Delta \circ \delta = p'$ and
  \[
    (\sigma \circ p_\Delta, q) \circ \delta =
    (\sigma \circ p_\Delta \circ \delta, q\delta) =
    (\sigma \circ p', q_\Gamma\sigma') =
    (p_\Gamma \circ \sigma', q_\Gamma\sigma') =
    (p_\Gamma, q_\Gamma) \circ \sigma' =
    1 \circ \sigma' =
    \sigma'
  \]

  For uniqueness assume morphism $\delta' \from \Delta' \to \Delta.A\sigma$
  such that $(\sigma \circ p_\Delta, q) \circ \delta' = \sigma'$ and $p_\Delta
  \circ \sigma' = p'$. As before we have $\delta' = (p_\Delta \circ \delta',
  q\delta') = (p', N)$ where $N \equiv q\delta'$. The following equalities
  hold:
  \[
    N = q\delta' = q_\Gamma(\sigma \circ p_\Delta \circ \delta', q\delta') = q_\Gamma((\sigma \circ p_\Delta, q) \circ \delta') = q_\Gamma\sigma' = M
  \]
  and therefore $\delta' = \delta$.
\end{answer}

\begin{exercise}
  Recall the notation $\int_\cat{C}\Gamma$ for the category of elements of a
  presheaf $\Gamma$ (section $1.2$ p.$16$ in Huber's thesis).  Check that
  substitution in $A\from (\int_\cat{C}\Gamma)^\op \to \Set$ with $\sigma \from
  \Delta \to \Gamma$ corresponds to precomposing with $\int_\cat{C} \sigma :
  \int_\cat{C} \Delta \to \int_\cat{C} \Gamma$ induced by $\sigma$.

  Then verify that this construction imposes a functor $\int_\cat{C} \from
  \mathit{Psh}(\cat{C}) \to \CAT$.
\end{exercise}
\begin{answer}
  Indeed, the operation of $\sigma$ on such objects is a precomposition
  satisfying the naturality axiom:
  \[
    \int_\cat{C}\sigma(I, \rho) := (I, \sigma_I \rho)
  \]
  where $\rho \in \Gamma(I), I \in \cat{C}$

  Let's check that $\int_\cat{C}$ takes us into a category. Recall the
  definition of $\int_\cat{C}\Gamma$:
  \begin{itemize}
    \item objects are pairs $(I, \rho)$ where $I \in \cat{C}$ and $\rho \in \Gamma(I)$
    \item a morphism $(J, \rho') \to (I, \rho)$ is a morphism $f \from J \to I$
      in $\cat{C}$ such that $\rho' = \rho f$.
  \end{itemize}

  We know our objects and morphisms, we get identities and composition by
  the equations $\rho 1 = \rho$, $(\rho f) g = \rho (f g)$ and the fact that
  morphisms $f$ are valid morphisms in the category $\cat{C}$.

  Let's now see that $\int_\cat{C}\sigma$ is a functor in $\CAT$. We have a
  mapping for objects $\int_\cat{C}\sigma(I, \rho)$ given above. The operation
  on arrows is as follows, where $f \from J \to I$, $g \from K \to J$:
  \[
    \int_\cat{C}\sigma(f) := \sigma_{I \to J} \Gamma_f
  \]
  Recall the naturality axiom for the natural transformation $\sigma$. It
  requires for each $I \in \cat{C}$ a corresponding $\sigma_I$. Hence the
  operation of $\sigma$ on the composition of morphisms in $\cat{C}$ can be
  considered as application of two indexed $\sigma_i$ of the corresponding
  objects between the morphisms (refer to the naturality square to see this).
  Below we give the indices on $\sigma$ that correspond to the place in the
  naturality square for clarity.  Then with associativity and keeping in mind
  that the composition $fg$ and the individual $f$ and $g$ are applied on the
  right we get $\int_\cat{C}\sigma(fg) = \sigma_{I \to K} \Gamma_{fg} =
  \sigma_{I \to K} \lambda \rho.\rho (fg) = \sigma_{I \to K} \lambda \rho.
  (\rho f)g = \sigma_{J \to K} (\sigma_{I \to J} \lambda \rho.\rho f) g =
  (\sigma_{J \to K} \lambda \rho.\rho g) (\sigma_{I \to J} \lambda \rho.\rho f)
  = (\sigma_{J \to K} \Gamma_g) (\sigma_{I \to J} \Gamma_f)$.  Which, together
  with $\rho 1 = \rho$ confirms that $\int_\cat{C}\sigma$ is indeed a functor.

  Thus, we conclude that $\int_\cat{C} \from \mathit{Psh}(\cat{C}) \to \CAT$ is
  a functor.
\end{answer}

\begin{exercise}
  From the slides we know that if $\Gamma \vdash A, B ~\text{type}$, i.e. $B$
  does not depend on $A$, then $\Pi x: A. B$ corresponds to the ordinary
  non-dependent function type:
  \begin{equation*}
    A \to B \vcentcolon= \Pi x: A. B
  \end{equation*}

  Give a term of type $A \to A$, and the elements in $\mathit{Ty}(\Gamma)$ and
  $\mathit{Ter}(\Gamma, A \to A)$ corresponding to the type and the term.

  \emph{Hint}: Read section $1.2.1$ in Huber's thesis for the general case.
\end{exercise}
\begin{answer}
  For a CwF $(\cat{C}, F), I, J \in ob(\cat{C}), f \from J \to I$ we give the
  element of $\mathit{Ty}(\Gamma)$ corresponding to the type $A \to A$:
  \[
    (A \to A)\rho = \{w | w_f u \in A(\rho f), u \in A(\rho f)\} = \{\lambda I f u.u\}
  \]
  The required equalities hold trivially for any $g \from K \to J$, $((\lambda
  I f u.u)_f u) g = (\lambda I f u.u)_{fg} (ug)$.

  $\Gamma, x : A \vdash x : A$ corresponds to $q$ in the model. Thus $\Gamma
  \vdash \lambda x.x : A -> A$ corresponds to $\lambda q$ being defined as
  follows:
  \[
    ((\lambda q) \rho)_f u = q(\rho f, u) = u
  \]
\end{answer}

\begin{exercise}
  Suppose we have two types $\Gamma \vdash A,B \text{ type}$ each not depending
  on the other.  Then we can define a product type in the presheaf model in the
  following way:
  \[
    (A \times B)\rho := \{(a,b) | a \in A\rho \land b \in B\rho\}
  \]

  Now suppose that we have dependent types $\Gamma \vdash A \text{ type}$ and
  $\Gamma, A \vdash B \text{ type}$.

  Define the product type that involves a dependent type (i.e. the dependent
  sum type) in the presheaf model.

  \[
    (\Sigma AB)\rho := ???
  \]

  \emph{Hint}: The definition is \emph{almost} identical to the one given
  above. But what do you need to change? To recall the presheaf model, read
  section $1.2$ in Huber's thesis.
\end{exercise}
\begin{answer}
  \[
    (\Sigma AB)\rho := \{(a,b) | a \in A\rho \land b \in B(\rho, a)\}
  \]
  where $(a, b)f = (af, bf)$ for $f \from J \to I$. The pairing operation
  $\Gamma \vdash (u, v) : \Sigma AB$ of $\Gamma \vdash u : A$ and $\Gamma.A
  \vdash v : B(1, u)$ is defined by the componentwise pairing: $(u, v)\rho =
  (u\rho, v\rho)$. Likewise for the projections $p$ and $q$, $(pw)\rho = a$ and
  $(qw)\rho = b$. This validates the necessary equations.
\end{answer}

\begin{exercise}
  Read Hofmann's section on the type of natural numbers (section 2.1.3) and
  familiarize yourself with identity types (ibid., section 2.1.5)

  By analogy to the type of natural numbers define the rules for a list type
  former which to any type $A$ associates a type $\mathit{List}(A)$ consisting
  of finite sequences of elements of $A$.

  \emph{Hint}: think of lists as inductively generated from the empty list by
  successive additions of elements of $A$ ("cons").

  Define a length function of type $\mathit{List}(A) \to \mathbb{N}$ and define
  a type $\mathit{Vec}_A(n)$ of lists of length $n$ for each $n : \mathbb{N}$
  using lists, the identity type and the $\Sigma$-type.
\end{exercise}
\begin{answer}
  First let's define the type formers:
  \[
    \inference[List-F]{\Gamma \vdash A ~\text{type}}{\Gamma \vdash \mathbf{List_A} ~\text{type}}\quad
    \inference[List-I-nil]{\Gamma \vdash A ~\text{type}}{\Gamma \vdash \mathit{nil_A} : \mathbf{List_A}}\quad
    \inference[List-I-cons]{\Gamma \vdash a : A \quad \Gamma \vdash x : \mathbf{List_A}}{\Gamma \vdash \mathit{Cons(a, x)} : \mathbf{List_A}}
  \]

  \[
    \inference[List-E]{
      \Gamma, l: \mathbf{List_A} \vdash B ~\text{type}\\
      \Gamma \vdash H_{nil} : B[nil_A/l]\\
      \Gamma, a : A, l : \mathbf{List_A}, x: B \vdash H_{cons} : B[Cons(a, l)/l]\\
      \Gamma \vdash M : \mathbf{List_A}
    }{\Gamma \vdash R^{\mathbf{List_A}}_{[l: \mathbf{List_A}]B}(H_{nil}, [a : A, l: \mathbf{List_A}, x : B] H_{cons}, M) : B[M/l]}
  \]

  \[
    \inference[List-C-nil]{
      \Gamma \vdash R^{\mathbf{List_A}}_{[l: \mathbf{List_A}]B}(H_{nil}, [a : A, l: \mathbf{List_A}, x: B] H_{cons}, nil) : B[nil/l]
    }{\Gamma \vdash R^{\mathbf{List_A}}_{[l: \mathbf{List_A}]B}(H_{nil}, [a : A, l: \mathbf{List_A}, x: B]H_{cons}, nil) = H_{nil} : B[nil/l]}
  \]

  \[
    \inference[List-C-cons]{
      \Gamma \vdash R^{\mathbf{List_A}}_{[l: \mathbf{List_A}]B}(H_{nil}, [a : A, l: \mathbf{List_A}, x: B] H_{cons}, Cons(N, M)) : B[Cons(N, M)/l]
    }{\Gamma \vdash R^{\mathbf{List_A}}_{[l: \mathbf{List_A}]B}(H_{nil}, [a : A, l: \mathbf{List_A}, x: B] H_{cons}, Cons(N, M)) = D}
  \]
  where $D = H_{cons}[N/a, M/l, R^{\mathbf{List_A}}_{[l: \mathbf{List_A}]B}(H_{nil}, [a : A, l: \mathbf{List_A}, x : B] H_{cons}, M)/x] : B[Cons(N, M)/l]$.

  Now we can define the length function:
  \[
    \mathit{Length(l) \vcentcolon= R^{\mathbf{List_A}}_{[l: \mathbf{List_A}]\mathbb{N}}(0, [a : A, l: \mathbf{List_A}, x : \mathbb{N}] Suc(x), l) : \mathbb{N}}
  \]

  And finally the $\mathit{Vec_A(n)}$ type:
  \[
    \mathit{Vec_A(n)} \vcentcolon= \Sigma l: \mathbf{List_A}. Id_\mathbb{N}(n, \mathit{Length}(l))
  \]

  Note that it is not the only way to define $\mathit{Vec_A(n)}$, as one could
  define it with the type formers analogously to our type $\mathit{List_A}$.
\end{answer}

\begin{exercise}
  Read the section on universes in Hofmann's notes (2.1.6) as well as the
  material for the previous exercise.

  Give the rules for universe $U$ containing a code for $\hat{0}$ for the empty
  type and a code $\hat{1}$ for the unit type $1$. Show that in a type theory
  which supports natural numbers, this universe, and the empty type itself the
  following type in the empty context is inhabited
  \[
    \diamond \vdash Id_\mathbb{N}(0, Suc(0)) \to 0 \text{ type}
  \]
  corresponding to Peano's fourth axiom $1 \neq 0$.

  \emph{Hint}: define using $R^\mathbb{N}$ a function $f \from \mathbb{N} \to
  U$ such that $\diamond \vdash f 0 = \hat{1}: U$ and $\diamond \vdash
  f(Suc(0)) = \hat{0} : U$
\end{exercise}
\begin{answer}
  \[
    \inference[U-0]{\vdash \Gamma~\text{ctxt}}{\Gamma \vdash \hat{0} : U}\quad
    \inference[U-0-Ty]{\Gamma \vdash \hat{0}: U}{\Gamma \vdash El(\hat{0}) = 0~\text{type}}
  \]

  \[
    \inference[U-1]{\vdash \Gamma~\text{ctxt}}{\Gamma \vdash \hat{1} : U}\quad
    \inference[U-1-T]{\Gamma \vdash \hat{1} : U}{\Gamma \vdash El(\hat{1}) = 1~\text{type}}
  \]

  \[
    f(n) = R^\mathbb{N}_{[n:\mathbb{N}]U}(\hat{1}, [n: \mathbb{N}, x: U]\hat{0}, n)
  \]

  Now to get an inhabitant of that type let's use $\mathit{Subst}$:
  \[
    P : Id(0, Suc(0)) \vdash \mathit{Subst}(P, (\star : El(f(0)))) : El(f(Suc(0)))
  \]
  Since the inhabitant to $El(f(0)) = El(\hat{1}) = 1~\text{type}$, namely
  $\star$, is always present by the formation rules, and $El(f(Suc(0))) =
  0~\text{type}$ we obtain an inhabitant to the required type.
\end{answer}

\end{document}

%%% Local Variables:
%%% mode: latex
%%% TeX-master: t
%%% End:
