\def\pathToRoot{../../}\input{\pathToRoot headers/uebungHeader}

\begin{document}
% Use Basis x or Talk x, where x is the number of the session
\uebunghead{Basis 1}{What is a Category?}

\author{Sarah Mameche, Andreas Meyer, Leonhard Staut}

\begin{hint}
  Read Chapter 1.1. For initial and terminal objects refer to Definition 2.17 on pages 48ff. For Monics and Epics, refer to the respective sections in Chapters 5.1 and 5.2.
\end{hint}



%\section{Diagrams}
%
%\begin{hint}
%  This is how you can draw diagrams:
%  \[
%    \begin{tikzcd}
%                                    & A \arrow{dr}{g}    & \\
%      B \arrow{ur}{f} \arrow{rr}{h} &                    & C
%    \end{tikzcd}
%  \]
%\end{hint}

\section{Categories - Definition and Basics}

\begin {exercise}
Show there can be at most one inverse for a morphism $f \from A \to B$.
\end{exercise}

\begin{answer}
  We have to show that for $f \from A \to B$, there is at most one $g
  \from B \to A$ s.t. $gf = 1_A$ and $fg = 1_B$.

  Assume that $g$ exists and that there is some $h \from B \to A$: $hf = 1_A$, $fh = 1_B$.

  We know from the identity axiom: $1_A \of g = g$ and $h \of 1_B = h$. Using this, our assumtions and the associativity axiom, we get:
 \[  1_A \of g = g \textiff (h \of f) \of g = g \textiff h \of (f \of g) = g \textiff h \of 1_B = g \textiff h=g. \]
\end{answer}

\begin {definition}{Rel}
The objects of the category \textbf{Rel} are sets. The morphisms $f \from A \to B$ are subsets $f \sub A \times B$.
The identity morphism on set $A$ is the equality relation $\{\round{a,a} \such a \in A\} := 1_A$.
Composition of two morphisms $f \sub A \times B$, $g \sub B \times C$ is defined as
\[ g \of f = \{ \round{a, c} \in A \times C \such \exists b. \round{a, b} \in f \land \round{b, c} \in g\} \]

\end{definition}

\begin {exercise}
Show that \textbf{Rel} is a category.
\end{exercise}


\begin{definition}{Pos} The objects of  \textbf{Pos} are partially ordered sets 
(recall that a poset is a set $A$ equipped with a reflexive, transitive and antisymmetric binary relation $\leq^A_\cdot $ ) 
An arrow $ m \from A \to B $ function that is monotone, i.e. $\forall a, a' \in A$, 
\[a \leq^A_\cdot a' \implies m(a) \leq^A_\cdot m(a')\]
\end{definition}

\begin{answer}
  textbf{identity laws:} for $f \sub A \times B$,
\[f \of 1_A = \{ \angles{a, b} \in A \times B \such \exists a' \emptybk (\angles {a, a'} \in 1_A \emptybk \& \emptybk \angles{a', b} \in f)\} = \{ \angles{a, b} \in A \times B \such \exists a' \in A.  \emptybk \angles{a', b} \in f)\} = f.\] Analogously, $1_B \of f = f$.


\textbf{associativity:} for $f \sub A \times B, g\sub B \times C, h \sub C \times D$,
\[
h \of (g \of f) = h \of \{ \angles{a, c} \in A \times C \such \exists b \emptybk (\angles {a, b} \in f \emptybk \& \emptybk \angles{b, c} \in g)\}
\]
\[= \{ \angles{a, d} \in A \times D \such \exists b, c \emptybk (\angles {a, b} \in f \emptybk \& \emptybk \angles{b, c} \in g \& \angles{c, d} \in h \} = (h \of g) \of f\]

\end{answer}


\begin {exercise}
Show that \textbf{Pos} is a category.
\end{exercise}


\begin{answer}
  \textbf{identity laws:} $1_A$ is the identity function, which is monotone since $a \leq^A_\cdot a' \implies a' \leq^A_\cdot a.$
\\The axiom holds because for $f \from A \in B, f \of 1_A = f = 1_B \of f$ (same as in \textbf{Set}).
\\
\textbf{associativity:} If $ f \from A \to B, g \from B \to C $ are monotone, so is $g \of f$ because \[a \leq^A_\cdot a' \implies f(a) \leq^A_\cdot f(a') \implies  g(f(a)) \leq^A_\cdot g(f(a'))  .\] Associativity follows from assiociativity of function composition.
\end{answer}

% \begin{exercise}
% For a fixed set X with Powerset $\mathcal{P}(X)$, does $\mathcal{P}(X) \cong \mathcal{P}(X)^{op}$ hold?
% \end{exercise}

\begin{definition}{Pointed Category} If an object is both \emph{initial} and \emph{terminal}, it is called a \emph{zero object}. A \emph{pointed category} is one with a zero object.
\end{definition}

\begin{exercise}
  \begin{enumerate}
  \item Show that \textbf{Rel} is a pointed category.
  \item Show that the category \textbf{Grp} of groups has both an initial and a final object, and that these are the same.
  \item Show that the category \textbf{Ring} of unital rings has both an initial and a final object, and that these are \emph{not} the same.
  \end{enumerate}
\end{exercise}

\begin{answer}
  \begin{enumerate}
  \item $\emptyset$ is the zero object. Since there is nothing, that could be related, the null relation is unique relation from $\emptyset$ to any other set S and vice versa.
  \item The trivial groups are zero objects.

    Initial oject:

    Let $\left({G, \circ}\right)$ be a group with identity element $e_G$ and $1={e}$ a trivial group, with operation $*: e*e := e$.

    Since group homomorphism preserves identity, any hypothetical group homomorphism $\phi: 1 \to G$ must satisfy:

    $\phi (e) = e_G$

    Show  that $\phi$ is  a group homomorphism:

    The initial object $1$ has only the element $e$,

    $\phi (e) \circ \phi (e) = e_G \circ e_G $ , Definition of $\phi$
    \\$= e_G$ , Definition of identity
    \\$= \phi (e)$
    \\$= \phi (e * e)$, Terminal object:

    Because singleton is terminal object on sets, there is precisely one mapping:

    $!: G \to \left\{{e}\right\}$

    defined by:

    $\forall g \in G: ! (g) = e$

    Show that $!$ is a group homomorphism.

    For any $g, h \in G$, we have:
    $!(g)*!(h) = e_G \circ e_G $ , Definition of !
    \\$= e$ , Definition of *
    \\$= !(g \circ h$ Defintion of !

  \item Initial is the ring of Integers $\mathbb{Z}$, terminal the zero ring consisting of only one element.

    Initial: Assume some $ \phi: \mathbb{Z} \mapsto R $, where R is a Ring. Since $\phi$ is a ring homomorphism, $\phi(n) = \phi(n*1) = n*\phi(1)$ holds and therefore, $\phi$ is unique.

    Terminal: Like in set, there is only the constant mapping between any ring and the one-element ring.
  \end{enumerate}
\end{answer}

\section{Monomorphisms and Epimorphisms}


\begin{exercise}
Show that any isomorphism is both monic and epic.
\end{exercise}

\begin{answer}
  Consider the diagram:
  \[
    \begin{tikzcd}
      A \arrow[r,shift left, "y"] \arrow[r,shift right,swap,"x"] &
      B \arrow[dr, shift left, "1"]  \arrow{r}{m}  &
      C \arrow{d}{e} \arrow[r,shift left, "i"] \arrow[r,shift right,swap,"j"] & D\\
     &  & B \arrow[ul, shift left, "1"]&
    \end{tikzcd}
  \]
  $m \from B \to C$ is an isomorphism with inverse $e \from C \to B$.
  Then:\\
  \begin{minipage}{.5\linewidth}
    \vspace{4mm}
    \centering $m$ is monic
    \[
      \begin{aligned}
        m \of x &= m \of y\\
        e \of (m \of x) &= e \of (m \of y)\\
        (e \of m) \of x &= (e \of m) \of y\\
        1_B \of x &= 1_B \of y\\
        x &= y
      \end{aligned}
    \]
  \end{minipage}%
  \begin{minipage}{.5\linewidth}
    \vspace{4mm}
    \centering $m$ is epic
    \[
      \begin{aligned}
        i \of m &= j \of m \\
        (i \of m) \of e &= (j \of m) \of e\\
        i \of (m \of e) &= j \of (m \of e)\\
        i \of 1_C &= j \of 1_C\\
        i &= j
      \end{aligned}
    \]

  \end{minipage}%
\end{answer}

\exercisesOnly{\newpage}

\begin{exercise}
  Let $A, B, C$ be objects in a category and let $f \from A \to B$ and $g \from B \to C$ be morphisms such that the following diagram commutes:
  \[
    \begin{tikzcd}
      A \arrow{r}{f} \arrow{dr}{h} & B \arrow{d}{g} & \\
      & C &
    \end{tikzcd}
  \]
  Show the following facts:
  \begin{enumerate}
  \item If both $f$ and $g$ are isomorphisms, then $h$ is also an isomorphism.
  \item If $h$ is monic, then $f$ is monic.
  \item If $h$ is epic, then $g$ is epic.
  \end{enumerate}
\end{exercise}

\begin{answer}
    \begin{itemize}
  \item If $f$ and $g$ are isomorphic, their inverses
    $f^{-1}$ and $g^{-1}$ exist.
    To show that $h = g \of f$ is also an isomorphism,
    we need to provide an inverse.\\

    \begin{minipage}{.5\linewidth}
      \[
      \begin{aligned}
        & \ \ \ \ (g \of f) \of (f^{-1} \of g^{-1})\\
        &= g \of (f \of f^{-1}) \of g^{-1} \\
        &= (g \of 1_B) \of g^{-1} \\
        &= g \of g^{-1} \\
        &= 1_C
      \end{aligned}
      \]
    \end{minipage}%
    \begin{minipage}{.5\linewidth}
      \[
      \begin{aligned}
        & \ \ \ \ (f^{-1} \of g^{-1}) \of (g \of f)\\
        &= f^{-1} \of (g^{-1} \of g) \of f\\
        &= (f^{-1} \of 1_B) \of f\\
        &= f^{-1} \of f\\
        &= 1_A
      \end{aligned}
      \]
    \end{minipage}%
    \vspace{4mm}
    \\
    $(f^{-1} \of g^{-1})$ is the inverse of $(g \of f)$.
    Therefore $h = (g \of f)$ is an isomorphism.
    \vspace{5mm}
  \item
    \raggedright Let $x, y \from D \to A$ be any morphisms
    from any object $D$ in the category. \\
    \centering
      \[
      \begin{aligned}
        f \of x &= f \of y \\
        g \of (f \of x) &= g \of (f \of y) \\
        (g \of f) \of x &= (g \of f) \of y \\
        x &= y \\
      \end{aligned}
      \]
  \item
    \raggedright Let $i, j \from C \to E$ be any morphisms
    to any object $E$ in the category.\\
    \centering
      \[
      \begin{aligned}
        i \of g &= j \of g \\
        (i \of g) \of f &= (j \of g) \of f \\
        i \of (g \of f) &= j \of (g \of f) \\
        i &= j \\
      \end{aligned}
      \]
  \end{itemize}
\end{answer}

\begin{exercise}
  Show the following for a morphism $f \from A \to B$ in the category \textbf{Set}:
  \begin{enumerate}
  \item If $f$ is monic, then it is injective.
  \item If $f$ is epic, then it is surjective. \textit{(Challenge 1)}
  \end{enumerate}
\end{exercise}

\begin{answer}
  \begin{itemize}
  \item Let $g,h \from \{x\} \to A$ two different functions
    from the one element set
    $\{x\}$ to $A$, with $g(x) = a_1$ and $g(x) = a_2$,
    for any $a_1,a_2 \in A$, with $a_1 \neq a_2$.\\
    We have: $g \neq h \xRightarrow{f \ mono} f \of g \neq f \of h$.
    Therefore: $f(a_1) = f(g(x)) \neq f(h(x)) = f(a_2)$.\\
    $\Rightarrow f$ is injective.
  \item Let $B$ be a two element set, e.g.
    $\{ \texttt{true}, \texttt{false}\}$.
    Let $g$ be the characteristic function of \texttt{Im}$(f)$,
    the image of f, which is defined by
    $\forall y \in Y. \ g(y) = \texttt{true}
    \Leftrightarrow y \in \texttt{Im}(f)$.
    And let $h$ be the constant \texttt{true}-function.
    $\forall y \in Y. \ h(y) = \texttt{true}$.
    Note that $g = h$, exactly if $f$ is surjective.\\
    We have: $g \of f = h \of f \xRightarrow{f \ epic} g = h$.\\
    $\Rightarrow$ f is surjective.
  \end{itemize}
\end{answer}

\begin{exercise}[Challenge 2]
  Show that in the category \textbf{Ring}, there are morphisms which are epic but not surjective.
  \begin{hint}
    Consider the inclusion morphism $\mathbb{Z} \to \mathbb{Q}$, and use the properties of ring homomorphisms.
  \end{hint}
\end{exercise}

\begin{answer}
  The inclusion map $f \from \mathbb{Z} \to \mathbb{Q}$ is defined by
  $f(x) = x$ and is clearly not surjective on $\mathbb{Q}$, because
  the $\texttt{Im}(f)$ doesn't contain any fractions like $\frac{1}{2}$.
  Let $g,h \from \mathbb{Q} \to \bf{R}$ be two morphisms from
  $\mathbb{Q}$ to any ring $\bf{R}$.
  $g,h$ agree on $\mathbb{Z}$, because there is exactly one ring
  homomorphism $\mathbb{Z} \to \bf{R}$ (In other words $\mathbb{Z}$
  is an initial object in the category \textbf{Ring}).
  Then we have $\forall p,q \in \mathbb{Z}, q \neq 0$:
  \[
    \begin{aligned}
      & \ \ \ \ g(\frac{p}{q})\\
      &= g(p) g(p^{-1})\\
      &= g(p) g(q)^{-1}\\
      &= h(p) h(q)^{-1}\\
      &= h(p) h(p^{-1})\\
      &= h(\frac{p}{q})
    \end{aligned}
  \]
  The first two steps are justified by the
  properties of ring homomorphisms to respect the structure.
  The third step uses the fact that $g$ and $h$ agree on
  $\mathbb{Z}$.\\
  Therefore: $g \of f = h \of f \Rightarrow g = h$, and $f$ is epic.
\end{answer}

\end{document}
%%% Local Variables:
%%% mode: latex
%%% TeX-master: t
%%% End:
