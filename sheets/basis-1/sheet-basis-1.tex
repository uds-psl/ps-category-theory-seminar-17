\def\pathToRoot{../../}\input{\pathToRoot headers/uebungHeader}

\begin{document}

% Use Basis x or Talk x, where x is the number of the session
\uebunghead{Basis 1}{What is a Category?}

\begin{hint}
  Read Chapter 1.1. For initial and terminal objects refer to Definition 2.17 on pages 48ff. For Monics and Epics, see the respective sections in Chapters 5.1 and 5.2.
\end{hint}

%\section{Diagrams}
%
%\begin{hint}
%  This is how you can draw diagrams:
%  \[
%    \begin{tikzcd}
%                                    & A \arrow{dr}{g}    & \\
%      B \arrow{ur}{f} \arrow{rr}{h} &                    & C
%    \end{tikzcd}
%  \]
%\end{hint}

\section{Categories - Definition and Basics}

\begin {exercise}
Show there can be at most one inverse for a map $f \from A \to B$.
\end{exercise}

\begin {definition}{Rel}
The objects of the category \textbf{Rel} are sets, an arrow $f \from A \to B$ is a subset $f \sub A \times B$. 
The identity arrow on set $A$ is the equality relation $\{\angles{a,a} \such a \in A\}$. 
Composition of two maps $f \sub A \times B$, $g \sub B \times C$ is defined as 
\[ g \of f = \{ \angles{a, c} \in A \times C \such \exists b \emptybk (\angles {a, b} \in f \emptybk \& \emptybk \angles{b, c} \in g)\} \]
\end{definition}

\begin {exercise}
Show that \textbf{Rel} is a category.
\end{exercise}


\begin{definition}{Pos} The objects of  \textbf{Pos} are partially ordered sets 
(recall that a poset is a set $A$ equipped with a reflexive, transitive and antisymmetric binary relation $\leq^A_\cdot $ ) 
An arrow $ m \from A \to B $ function that is monotone, i.e. $\forall a, a' \in A$, 
\[a \leq^A_\cdot a' \implies m(a) \leq^A_\cdot m(a')\]
\end{definition}

\begin {exercise}
Show that \textbf{Pos} is a category.
\end{exercise}

\end{document}

%%% Local Variables:
%%% mode: latex
%%% TeX-master: t
%%% End:
