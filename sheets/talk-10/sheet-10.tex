\def\pathToRoot{../../}\documentclass{article}

\usepackage{nag}
\makeatletter
\@ifclassloaded{beamer}{}
{\usepackage[small,compact]{titlesec}}
\makeatother
\usepackage[utf8]{inputenc}
\usepackage[T1]{fontenc}
\usepackage{lmodern}
\usepackage{color}
\usepackage{parskip}
\usepackage{needspace}
\usepackage{microtype}
\usepackage{mathtools}
\usepackage{xifthen}
\usepackage{xpatch}
\usepackage{enumitem}
\usepackage{mdwlist}
\usepackage{bussproofs}
\EnableBpAbbreviations
\usepackage{tabu}
\usepackage{amssymb}
\usepackage{amsmath}
\usepackage{amsthm}
%grober hack, der den groben hack von parskip bei den amsthm sachen korrigiert
\begingroup
    \makeatletter
       \@for\theoremstyle:=definition,remark,plain\do{%
            \expandafter\g@addto@macro\csname th@\theoremstyle\endcsname{%
                        \addtolength\thm@preskip\parskip
             }%
        }
\endgroup
\usepackage[UKenglish]{babel}
\usepackage{xparse}
\usepackage{adjustbox}
\usepackage{geometry}
\usepackage{booktabs}
\usepackage{multicol}
\usepackage{soul}
\usepackage{calc}
\usepackage{textcase}
\usepackage{stmaryrd}
\usepackage{marvosym}
\usepackage{wasysym}
\usepackage{pifont}
\newcommand{\cmark}{\ding{51}}
\newcommand{\xmark}{\ding{55}}
\usepackage{tikz}
\usetikzlibrary{trees, backgrounds, shapes, chains, decorations.text, decorations.pathreplacing, circuits.logic.IEC, patterns, matrix}
\usepackage{tikz-qtree}
\usepackage{tikzsymbols}
\usepackage{fancyvrb}
\usepackage{fancyhdr}
\usepackage{verbatim}
\usepackage[framemethod=tikz]{mdframed}
\usepackage{lastpage}
\usepackage{pgfpages}
\usepackage{csquotes}
\usepackage{longtable}
\usepackage{ragged2e}
%\usepackage{stackengine}
\usepackage{censor}
\usepackage{expl3}
\usepackage{multirow}
\usepackage{hyperref}
\usepackage{environ}


% Package for Cateogry diagrams:

\usepackage{tikz-cd}



\ifcsdef{labelenumi}{
\renewcommand{\labelenumi}{(\alph{enumi})}
\renewcommand{\labelenumii}{(\roman{enumii})}
}{}

\input{\pathToRoot headers/definitions}



\tikzset{
    normal/.style={draw, semithick},
    n/.style={style=normal, circle, inner sep=1mm, minimum size=8mm},
    l/.style={style=normal, rounded corners=1mm, inner sep=1mm, minimum size=6mm},
    e/.style={style=normal, shorten >=1mm, shorten <=1mm, ->, >=stealth},
    syntax/.style={style=normal, ellipse, minimum height=6mm, minimum width=8mm}, % nodes in syntax trees
    inner/.style={style=normal, minimum size=4mm}, % inner leaves or root in normal trees
    leaf/.style={style=normal, circle, minimum size=4mm}, % leaves in normal trees
    te/.style={style=normal}, % edges in a tree
    be/.style={style=e, dashed} % binding edge
}

\newcommand{\syntaxtree}[1]{ % DEPRECATED - use tikzsyntaxtree
    \begin{tikzpicture}[baseline=(current bounding box.north)]
        \tikzset{grow=down}
        \tikzset{every node/.style={syntax}}
        \tikzset{edge from parent/.style=
            {te,
                edge from parent path={(\tikzparentnode) -- (\tikzchildnode)}}}
        \Tree #1
    \end{tikzpicture}
}

\newenvironment{tikzsyntaxtree}[1][]{
    \begin{tikzpicture}[baseline=(current bounding box.north), #1]
    \tikzset{grow=down}
    \tikzset{every tree node/.style={syntax}}
    \tikzset{edge from parent/.style={te, edge from parent path={(\tikzparentnode) -- (\tikzchildnode)}}}
}{
    \end{tikzpicture}
}


\newcommand{\DisplayScaledProof}{\maxsizebox{\linewidth}{!}{\DisplayProof}}
\newcommand{\DisplayTopProof}{\adjustbox{valign=t}{\DisplayProof}}
\newcommand{\DisplayScaledTopProof}{\adjustbox{valign=t}{\maxsizebox{\linewidth}{!}{\DisplayProof}}}


\newcolumntype{P}[1]{>{\RaggedRight\hspace{0pt}}p{#1}}

\newenvironment{prooftable}
{
    \begin{longtable}{>{\footnotesize}p{0.33\textwidth}>{\footnotesize}p{0.33\textwidth}|>{\footnotesize}P{0.15\textwidth}}
    \normalsize Textbeweis & \normalsize Erklärungen & \normalsize Schlussregel\\\hline
    \endhead
}
{
    \end{longtable}
}


\theoremstyle{definition}
\newtheorem*{definition*}{Definition} % Definition ohne Nummer
\newtheorem*{inferenceRule*}{Schlussregel}

\usepackage{titling}
\geometry{a4paper,left=2cm,right=2cm,top=2cm,bottom=3cm}


\newcommand{\licenseccjuliachristian}{\def\islicenseccjuliachristian{}}
\newcommand{\suppresslicense}{\def\issuppresslicense{}}


\AtBeginDocument{
    \pagestyle{fancy}
    \renewcommand{\headrulewidth}{0pt}
    \renewcommand{\footrulewidth}{1pt}
    \fancyhead{}
    \fancyfoot[C]{\thepage~/~\pageref{LastPage}}
    \fancyfoot[R]{\footnotesize exercise sheet from \\ \theauthor}

}


\newcommand{\pgbreakhere}{\Needspace*{4\baselineskip}}
\newcommand{\pgbreakHere}{\Needspace*{10\baselineskip}}
\newcommand{\pgbreakHERE}{\Needspace*{15\baselineskip}}

\newcommand{\raisedrule}[2][0em]{\leavevmode\leaders\hbox{\rule[#1]{1pt}{#2}}\hfill\kern0pt}

% inspired by http://tex.stackexchange.com/questions/242294/suppress-parskip-only-after-a-specific-paragraph
\makeatletter
\newlength\noparskip@parskip % used to store a backup of the parskip value
\newboolean{noparskip@triggered} % flag to indicate that noparskip was run in the current paragraph
\setboolean{noparskip@triggered}{false}
\newboolean{noparskip@active} % flag to indicate that parskip should be restored after this paragraph
\setboolean{noparskip@active}{false}
\let\noparskip@par\par % store a backup of the \par command
\@setpar{% redefine \par with the means of ltpar.dtx to stay compatible to enumerate and itemize
    \ifhmode% since we're counting occurrences of \par, \par\par would be a problem, so check that we are actually ending a paragraph
        \ifthenelse{\boolean{noparskip@active}}{%
            \setlength\parskip\noparskip@parskip% restore parskip
            \setboolean{noparskip@active}{false}% remember not the restore parskip again
        }{}%
        \ifthenelse{\boolean{noparskip@triggered}}{%
            \ifthenelse{\boolean{noparskip@active}}{}{
                % we are triggering noparskip and not currently in a noparskip already
                \setlength\noparskip@parskip\parskip % copy the current parskip into the backup variable
            }%
            \setboolean{noparskip@triggered}{false}% paragraph is ending, so noparskip is no longer triggered
            \setlength\parskip{0pt}% no parskip when the next paragraph begins
            \setboolean{noparskip@active}{true}% parskip must be restored by the next par
        }{}%
    \fi%
    \noparskip@par% run the original par command
}
\def\noparskip@backout{%
    \ifthenelse{\boolean{noparskip@active}}{%
        % a list is beginning and parskip is currently set to zero, wich would mess up the list
        \setlength\parskip{\noparskip@parskip}% restore parskip before the list begins
        \setboolean{noparskip@active}{false}%
    }{}%
    \setboolean{noparskip@triggered}{false}% there's no sense in keeping noparskip triggered throughout a list
}
\xpretocmd\begin{%
    \ifstrequal{#1}{enumerate}{\noparskip@backout}{}%
    \ifstrequal{#1}{itemize}{\noparskip@backout}{}%
    \ifstrequal{#1}{list}{\noparskip@backout}{}%
    \ifstrequal{#1}{proof}{\noparskip@backout}{}%
}{}{}
\def\noparskip{%
    \leavevmode% ensure that we are within a paragraph
    \setboolean{noparskip@triggered}{true}% trigger noparskip
}
\makeatother

\newcommand{\noparskipworkaround}{} % DEPRECATED and no longer needed


\newcommand{\head}[1]{
    {
        \setlength{\parskip}{0pt}
        \hrule height 1pt
        \vspace{.2cm}
        Saarland University \hfill Category Theory Seminar 2017\par
        Programming Systems Lab \hfill \small\url{https://courses.ps.uni-saarland.de/ct_ss17/}\par
        \tiny\raisedrule[0mm]{1pt}
        \vspace{2ex}
        \begin{center}
            \Large
            \textbf{#1}\par
            \raisedrule[2mm]{1pt}
        \end{center}
        \vspace{3ex}
    }
}

\newenvironment{leftframedparagraph}{\begin{mdframed}[hidealllines = true, leftline = true, innerleftmargin = 2ex, innerrightmargin = 0pt,
innertopmargin = 0pt, innerbottommargin = 2pt, skipabove=2ex, skipbelow=1ex, outerlinewidth = 0ex, innerlinewidth = 0.5ex]}{\end{mdframed}}
\newenvironment{leftframed}{\begin{mdframed}[hidealllines = true, leftline = true, innerleftmargin = 2ex, innerrightmargin = 0pt,
innertopmargin = 0pt, innerbottommargin = 0pt, skipabove=2ex, skipbelow=1ex, outerlinewidth = 0ex, innerlinewidth = 0.5ex]}{\end{mdframed}}

%%% Local Variables:
%%% mode: latex
%%% TeX-master: t
%%% End:


\newcommand{\uebunghead}[3][Exercise sheet:]{\def\sheetid{#2}\head{#1 #2\ifthenelse{\isundefined{\issolution}}{}{ \ifthenelse{\isundefined{\ismarking}}{(Possible solutions)}{(Marking)}} \\ #3}}

\licenseccjuliachristian


\newcommand{\amountofpoints}[1]{\ifstrequal{#1}{1}{1~Punkt}{#1~Punkte}}


% marking implies solution
\ifthenelse{\isundefined{\ismarking}}{}{\def\issolution{}}


%%%Environments
\newcounter{ExamExerciseCounter} % will only be used in exams, but must be defined here so ExerciseCounter can be reset when ExamExericise counts
\setcounter{ExamExerciseCounter}{0}
\newcounter{ExerciseCounter}[ExamExerciseCounter]
\setcounter{ExerciseCounter}{0}

\newcommand{\ExerciseNumber}{\sheetid.\arabic{ExerciseCounter}}
\renewcommand{\theExerciseCounter}{\ExerciseNumber}

\newcommand{\ExercisePointHook}[1]{}

%Aufgaben-Umgebung
\NewDocumentEnvironment{exercise}{od<>}{
    \refstepcounter{ExerciseCounter}
    \pgbreakhere
    \vspace{1ex}\textbf{Exercise\ \ExerciseNumber}%
    \IfNoValueF{#1}{ \emph{(#1)}}%
    \IfNoValueF{#2}{\hfill(\amountofpoints{#2})}%
    \IfNoValueF{#2}{\ExercisePointHook{#2}}%
    \noparskip\par\nopagebreak
}{
    \par
    \vspace{2ex}
}

\newcommand{\exercisesOnly}[1]{\ifthenelse{\isundefined{\issolution}}{#1}{}}

%Loesungs-Umgebung
\newenvironment{answer}
{
    \ifthenelse{\isundefined{\issolution}}
    {
        \comment
    }{
        \vspace{1ex}\textsl{Sample solution \ExerciseNumber}\noparskip\par\nopagebreak
    }
}{
    \ifthenelse{\isundefined{\issolution}}
    {
    }{
        \vspace{1ex}
        \hspace*{\fill}
    }
}

\newenvironment{marking}
{%
    \ifthenelse{\isundefined{\ismarking}}%
    {%
        \comment%
    }{%
        \color{red}
    }%
}{%
    \ifthenelse{\isundefined{\ismarking}}%
    {%
    }{%
    }%
}

\newenvironment{example}{\begin{leftframedparagraph}\paragraph{Example:}}{\end{leftframedparagraph}}
\newenvironment{hint}{\paragraph{Hint:}}{}
\newenvironment{caution}{\paragraph{Caution:}}{}
\newenvironment{definition}[1]{\begin{leftframedparagraph}\paragraph{Definition (#1):}}{\end{leftframedparagraph}}


\begin{document}

% Use Basis x or Talk x, where x is the number of the session
\uebunghead{Talk 10}{Presheaves, Representables and the Yoneda Lemma}
\author{Dominik Wagner}

\begin{hint}
  Read chapter 4 in Leinster and chapter 8 in Awodey.
\end{hint}

\begin{definition}{Right $M$-Sets}
  Let $M$ be a monoid. 
  \begin{itemize}
  \item A \emph{right $M$-set} is a set $S$ together with a function
    \begin{align*}
      \cdot\from S\times M&\to S\\
      (s,m)&\mapsto s\cdot m
    \end{align*}
    such that $s\cdot (mm')=(s\cdot m)\cdot m'$ and $s\cdot 1_M=s$ for all $m,m'\in M$ and $s\in S$.
    \item A homomorphism between two right $M$-sets $S_1$ and $S_2$ is a mapping $\alpha\from S_1\to S_2$ satisfying $\alpha(s\cdot_{S_1} m)=\alpha(s)\cdot_{S_2}m$.
    \item The \emph{right regular representation} $\underline{M}$ is the $M$-set $M$ together with the right-operation $s\cdot m=sm$, i.e. the underlying monoid-operation.
  \end{itemize}

\end{definition}

\begin{exercise}[Not difficult despite its length]
  In this exercise we prove the Yoneda Lemma for the special case of one-object categories, i.e. monoids regarded as categories.
  We exploit the correspondence between presheaves on one-object categories (monoids $M$) and right $M$-sets.

  \begin{enumerate}
  \item Make yourself familiar again with this correspondence (e.g. by reading the examples 1.2.14 and 1.2.8 in Leinster).
  \item Let $M$ be a monoid. Show that the $M$-set corresponding to the unique(!) representable functor $M^\op\to \Set$ is the right regular representation $\underline M$.
  \item Now let $S$ be any right $M$-set. Show that for each $s\in S$ there is a unique homomorphism $\alpha\from\underline M\to S$ of $M$-sets such that $\alpha(1)=s$. Deduce that 
    \begin{align*}
      \{\text{homomorphisms }\underline M\to S\}\iso S
    \end{align*}
  \item Deduce that
    \begin{align*}
      [M^\op,\Set](H_A, F)\iso F(A)
    \end{align*}
    holds for all presheaves $F\from M^\op\to\Set$.
  \end{enumerate}
\end{exercise}

\begin{answer}
  \begin{enumerate}
  \item Done!
  \item Let $A$ be the unique object corresponding to $M$. In what follows we identify elements $m$ of the monoid $M$ with arrows $A\xrightarrow{m}A$. Then, $H_A(A)=\{\text{maps } A\to A\}= M$. Furthermore, for all $m,m'\in M$, $m\cdot m'=H_A(m')(m)=m\of m'=mm'$. Hence, $H_A$ contains exactly the same information as $\underline M$.
  \item Let $S$ be a right $M$-set. Assume there is a homomorphism $\alpha\from \underline M\to S$. Then for each $s\in S$, 
    \begin{align*}
      \label{eq:defa}
      \alpha(s)=\alpha(1\cdot_M s)=\alpha(1)\cdot_S s.
    \end{align*}
    Thus, $\alpha$ is completely defined by its action on $1$ if it exists. At the same time, this equation completely defines a homomorphism since
    \begin{align*}
      \alpha(s)\cdot m=(\alpha(1)\cdot s)\cdot m=\alpha(1)\cdot(s\cdot m)=\alpha(s\cdot m)
    \end{align*}
    as required.
    Therefore, the function
    \begin{align*}
      \{\text{homomorphisms }\underline M\to S\}&\to S\\
      \alpha&\mapsto \alpha(1)
    \end{align*}
    is an isomorphism.
  \item To conclude the Yoneda lemma we still have to show that natural transformations $F_1\to F_2$ are the same as homomorphisms between $S_1$ and $S_2$ where $S_1,S_2$ are the $M$-sets corresponding to the presheaves $F_1$ and $F_2$. 

    To see this let us instantiate the definition of naturality for our special case. It states that for each arrow $A\xrightarrow{m}A$ the following diagram commutes:

    \begin{align*}
      \xymatrix{
      F_1(A)=S_1 \ar[r]^{F_1(m)=-\cdot m} \ar[d]^{\alpha} & F_1(A)=S_1 \ar[d]_{\alpha} \\
      F_2(A)=S_2 \ar[r]^{F_2(m)=-\cdot m} & F_2(A)=S_2
                   }
    \end{align*}
    (since there is only one object) or more concretely for each $s\in S_1$,
    \begin{align*}
      \alpha(s\cdot m)=\alpha(s)\cdot m.
    \end{align*}

    Now the Yoneda lemma (for our special case) is an easy consequence:
    \begin{align*}
      [M^\op,\Set](H_A, F)\iso \{\text{homomorphisms }\underline M\to S\}\iso S= F(A)
    \end{align*}
    where $S$ is the $M$-Set corresponding to $F$.
  \end{enumerate}
\end{answer}

% \begin{exercise}
%   A presheaf X:C op→SetX : C^{op} \to Set is representable precisely if the comma category (Y,const X)(Y,const_X) has a terminal object. If a terminal object is (d,f:Y(d)→X)≃(d,f∈X(d))(d, f : Y(d) \to X) \simeq (d, f \in X(d)) then X≃Y(d)X \simeq Y(d).

% This follows from unwrapping the definition of morphisms in the comma category (Y,const X)(Y,const_X) and applying the Yoneda lemma to find
% (Y,const X)((c,f∈X(c)),(d,g∈X(d)))≃{u∈C(c,d):X(u)(g)=f}. (Y,const_X)((c,f \in X(c)), (d, g \in X(d))) \simeq \{ u \in C(c,d) : X(u)(g) = f \} \,.

% Hence (Y,const X)((c,f∈X(c),(d,g∈X(d)))≃pt(Y,const_X)((c,f \in X(c), (d, g \in X(d))) \simeq pt says precisely that X(−)(f):C(c,d)→X(c)X(-)(f) : C(c,d) \to X(c) is a bijection.
% \end{exercise}


\end{document}

%%% Local Variables:
%%% mode: latex
%%% TeX-master: t
%%% End:
